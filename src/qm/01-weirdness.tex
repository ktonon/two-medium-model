\subsection{Why ``quantum weirdness'' shows up}

In 2MM, every form of detectable matter---including the apparatus doing
the detecting---is a \textbf{composite standing-wave pattern} in the
LCM. When you push measurement down toward the characteristic wavelength
of that medium:

\begin{itemize}
\tightlist
\item
  The structure of the LCM becomes elastic, deformable, and dominated by
  interference.
\item
  Measurement tools cannot anchor themselves to anything ``rigid''
  because both the target and the probe are excitations of the same
  substrate.
\item
  The stability conditions of LCM waves enforce intrinsic
  bandwidth/location tradeoffs---an analogue of uncertainty, not because
  of fundamental metaphysics, but because of \textbf{LCM-wave   mechanics}.
\end{itemize}

Thus Heisenberg uncertainty emerges as a \textbf{structural property of our medium}, not a universal principle of existence.
