\subsection{The Locality Problem and Photon Collapse}

One of the strangest questions in modern physics is this:
\emph{When a photon spreads out as a wave and then hits a detector, how does all of its energy end up in one tiny spot?}
If its light spreads in all
directions, why doesn't the energy smear out everywhere? And when the
photon shows up at a single point, why does every other place it could
have landed instantly become irrelevant? People often call this a
problem of ``non-locality,'' but the heart of the puzzle is simpler:
\textbf{How does the photon know where to go, when nothing in the wave spreading outward seems to tell it?}

The Two-Medium Model offers a way to understand this that feels physical
rather than magical. And it begins long before the photon arrives.

\paragraph{Setting the Stage: The GCM Shapes the Terrain}

Every piece of matter---every atom in a detector, every grain of the
screen---casts tiny shadows in the deeper GCM medium. These shadows
spread with unimaginable speed and sculpt a kind of invisible terrain
throughout the surrounding space. Hills, valleys, channels,
pockets---all defined by how matter distorts the GCM flux.

From the viewpoint of the slower LCM, this terrain is simply
\emph{there}---settled long before the photon enters the experiment.

\paragraph{The Photon Enters the Landscape}

What we call a photon is a small, structured packet traveling inside a
much larger ripple in the LCM. The ripple expands widely, but the packet
itself remains compact. It doesn't fly outward in all directions---it
follows a path defined by the LCM's shape, which in turn is shaped by
the GCM terrain.

So as the photon moves toward the detector, it isn't guessing where it
will end up. It is moving through a landscape that has already been
carved by the matter ahead of it.

\paragraph{Capture Zones and Collapse}

In certain places near the detector, the GCM shadows create steep
gradients and compress the LCM so strongly that the photon's packet
becomes unstable. These regions behave like ``capture zones,'' places
where the photon cannot continue as a free wave. When the packet enters
such a pocket, it breaks apart and transfers its energy to a local
electron or atomic structure.

Importantly, \textbf{only one of these capture zones matches the photon's structure at any given moment.} The rest of the illuminated
space---though reached by the ripple---was never a true candidate for
collapse. Those places had no GCM-created pocket waiting to absorb the
packet.

\paragraph{Why Collapse Looks Instantaneous}

To us, bound to the slow-moving LCM, it seems as though a photon spread
everywhere suddenly chooses one point in an instant. But in 2MM, nothing
spreads everywhere except the ripple. The packet remains localized and
simply follows the terrain. The moment it enters a capture zone,
collapse happens. No faster-than-light signal is needed because nothing
elsewhere ever had the packet to begin with.

The appearance of ``instantaneous collapse'' is really just the moment
when a photon falls into a trap that the universe prepared in advance.

\begin{quote}
\textbf{Acknowledgment: Paul Marmet}

It was Paul Marmet's book \emph{Absurdities in Modern Physics: A Solution}
that helped me understand the problem of locality.
Marmet emphasized a fundamental puzzle: how can the energy of a
radially expanding wavefront collapse into a single point upon
detection? His proposed resolution---considering the reference frame of
the light wave itself, for which vast spatial separations contract and
exchanges across light-years become instantaneous---was the first
conceptual approach I encountered that truly confronted the paradox
head-on.

In the context of the Two-Medium Model, this insight takes on a new
interpretation. The connection is not instantaneous, nor does it require
shifting to the ``frame of the photon,'' but arises from the influence
of the ultra-fast GCM shaping the collapse conditions far in advance.
Marmet's work was an early signal that locality, as usually understood,
is insufficient---and that a deeper, layered medium might be necessary
to make sense of how the universe actually behaves.
\end{quote}
