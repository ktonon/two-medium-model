\subsection{How 2MM Demystifies Quantum Behavior}

The photon's collapse need not be framed as a mysterious choice or a
global coordination problem. In the Two-Medium Model, it becomes the
natural consequence of a deeper structure shaping the visible world. The
GCM, operating at scales and speeds far beyond the LCM, quietly carves
the landscape in which quantum events unfold. Matter shapes this hidden
terrain in advance, and the photon's packet simply follows its contours.
The ripple may spread broadly, but the packet travels within a
pre-sculpted channel, and the point of collapse is determined long
before the photon arrives. The universe, in this view, behaves not like
a gambler rolling dice, but like a valley guiding a traveler toward a
single, inevitable destination.

This behavior---one localized event emerging from an apparently
delocalized wave---is not merely compatible with 2MM; it is one of the
strongest indicators that a deeper medium must exist. The LCM cannot
update itself instantaneously, yet collapse appears instantaneous.
Rather than invoking nonlocal magic, 2MM offers a simpler explanation:
the ultrafast GCM establishes the collapse geometry ahead of time. What
looks like a paradox is actually a visible trace of this underlying
substrate doing its work.

What 2MM provides, then, is not a substitute for quantum mechanics, but
a framework for understanding why quantum mechanics looks the way it
does. Once we accept that the LCM is only one layer of a multi-medium
universe, the familiar puzzles of locality, collapse, and ``weirdness''
no longer demand special interpretation. They fall naturally out of the
interplay between a slow, wave-based medium and a deeper, faster one
that shapes it. With that perspective in hand, the reader is free to
reinterpret quantum experiments independently, using 2MM as a guide
rather than a doctrine. The demystification comes not from new
mathematics, but from a clearer sense of what the underlying
architecture of reality might be.
