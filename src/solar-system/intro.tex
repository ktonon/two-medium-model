The idea that Earth's radius may have changed over geological time has
appeared in many forms throughout the history of geophysics. Although it
sits outside the modern mainstream, it has been examined seriously by
researchers such as Samuel Warren Carey, who viewed expansion as a
possible unifying framework for continental drift, and Dr. James Maxlow,
who produced detailed reconstructions suggesting smaller ancient Earth
radii. A full discussion of the evidence and counter-arguments is beyond
the scope of this paper; readers interested in the broader debate are
encouraged to consult those primary works.

What matters for 2MM is not the historical argument itself, but the
striking fact that the model approaches the possibility of planetary
growth from a completely different direction. Rather than beginning with
geological interpretation, 2MM arrives at the idea through the internal
physics of the LCM. Under sustained high compression, the model predicts
that LCM-wave collapse and proton formation act as a continuous energy
sink in planetary interiors, naturally introducing slow mass increase
over geological timescales.

One of the long-standing obstacles for expansion hypotheses has been the
absence of a physically grounded mechanism capable of driving sustained
planetary growth. 2MM does not claim to resolve the entire geological
debate, but it does offer a plausible mechanism where none previously
existed---providing a fresh context in which expansion models can be
reconsidered.
