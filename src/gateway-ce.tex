\input{preamble}
\usepackage[style=numeric, sorting=none]{biblatex}
\addbibresource{references.bib}

\title{%
Gravity, Energy, and Cosmological Equilibrium
}

\author{Kevin Tonon}
\date{\today}

\begin{document}
\maketitle

\begin{abstract}
Modern cosmology is largely organized around metric expansion as its primary explanatory principle. While empirically successful, this framing leaves limited room for equilibrium-based reasoning, despite the prominence of equilibrium concepts across much of physics. This note explores cosmology from an alternative perspective centered on dynamic balance: between localized structure and diffuse background conditions, clustered matter and vast cosmic voids, and localized versus distributed forms of energy. Gravitational dynamics are central to all of these phenomena, yet are typically treated primarily as a geometric descriptor rather than as an explicit energy-transfer process. When gravity is considered in terms of work and energy transfer, questions of large-scale consistency and balance naturally arise. This note argues that equilibrium-based considerations may be essential to any fully satisfactory cosmological framework.
\end{abstract}

Contemporary cosmology has achieved remarkable success by interpreting a wide range of observations through the lens of an expanding spacetime. Within this framework, redshift, large-scale structure, and background radiation are coherently described by a single organizing principle. For many purposes, this description is sufficient.

At the same time, reliance on expansion as the dominant explanatory framework constrains the kinds of questions that are asked. In particular, cosmology makes comparatively limited use of \textbf{equilibrium-based reasoning}, despite its central role across much of physics. Many familiar systems are governed not by one-way evolution alone, but by ongoing balances between competing processes.

This motivates a broader question: \textit{might large-scale cosmological structure reflect dynamic equilibrium rather than irreversible expansion?}

The universe exhibits strong localization of matter and energy into compact structures, while the overwhelming majority of its volume resides in low-density regions. Clustering and void dominance coexist across scales. These features suggest compensating processes operating simultaneously, rather than a single monotonic driver.

Gravity is central to this organization. It governs the formation of structure and the motion of matter, and plays a dominant role across all scales. In practice, gravitational models reproduce observed motions only by introducing additional components whose physical origin remains unclear, indicating that questions of energy flow, partitioning, and long-term balance are not yet fully resolved.

Gravity performs mechanical work. Bodies accelerating under gravity gain kinetic energy, and bound systems exchange energy continuously. When gravitational dynamics are viewed through the lens of work and energy transfer, further questions arise: how energy is partitioned between localized and diffuse states, how dense regions relate energetically to surrounding voids, and how large-scale balance is maintained over time.

These considerations motivate an alternative way of organizing cosmological phenomena—one centered on \textbf{dynamic equilibrium} rather than global expansion. In such a view, large-scale structure reflects ongoing balances between confinement and dispersal, clustering and compensation, and local organization and global consistency.

This note does not attempt to provide a complete cosmological model. Its purpose is to motivate equilibrium as a serious and underexplored organizing principle, and to clarify the kinds of questions that emerge once such a perspective is adopted. The accompanying paper develops these ideas into a coherent conceptual framework and presents qualitative predictions spanning particle physics, planetary interiors, and large-scale structure, while addressing consistency with global energy conservation \cite{tonon2mm}.

\printbibliography

\end{document}
