\input{preamble}
\usepackage[style=numeric, sorting=none]{biblatex}
\addbibresource{references.bib}

\title{%
Planetary Interiors as Evolving Systems
}

\author{Kevin Tonon}
\date{\today}

\begin{document}
\maketitle

\begin{abstract}
Planetary science increasingly confronts a set of recurring observations that resist unification under standard formation-and-cooling models. Persistent internal activity, long-lived volcanism, excess emitted energy in giant planets, widespread subsurface oceans, volatile-rich interiors, and large-scale structural asymmetries suggest that many planetary bodies are not chemically or energetically static remnants of formation. These phenomena are typically addressed through body-specific mechanisms, yet their repetition across terrestrial planets, gas giants, and large moons points to a missing organizing principle governing long-term internal evolution. This note highlights the common pattern underlying these observations and motivates the need for a unifying physical framework capable of accounting for sustained internal energy, volatile persistence, and structural change over geological timescales.
\end{abstract}

\section{Motivation}

The prevailing view in planetary science treats planets and moons as systems that form hot, differentiate early, and then cool and evolve through the gradual release of stored energy \cite[p.~4]{TurcotteSchubert2014}. Radiogenic heating, secular cooling, and—in some cases—tidal dissipation are expected to dominate long-term behavior. For many individual bodies, this framework provides workable explanations.

Taken across the Solar System, however, a broader pattern emerges. Planetary bodies of very different sizes, compositions, and orbital environments exhibit internal activity that persists far longer than simple cooling models would suggest. These behaviors are commonly explained through specialized, object-specific processes. When viewed collectively, they point toward a deeper commonality that remains underexplored.

\section{Persistent Internal Energy}

Several classes of planetary bodies exhibit internal energy that does not diminish in the expected manner. Gas giants typically emit more energy than they absorb from the Sun, a fact attributed to slow gravitational contraction and internal differentiation, though with notable inconsistencies across planets. Terrestrial bodies such as Earth maintain long-lived volcanism and tectonic activity over billions of years. Smaller bodies, including Io and several icy moons, display intense volcanism, subsurface oceans, and active resurfacing despite limited radiogenic heat and minimal solar input.

These are not identical phenomena, but they share a common feature: \textbf{internal energy remains dynamically important far longer than passive cooling alone would predict}. Existing explanations can be made to fit individual cases, but they do not form a unified or predictive account across planetary classes.

\section{Volatile Persistence and Mobility}

A related pattern appears in the behavior of volatiles. Hydrogen-bearing compounds—most notably water—play a central role in planetary interiors and surfaces across the Solar System. Earth maintains a substantial and persistent internal water inventory. Multiple icy moons host long-lived subsurface oceans. Volatile-driven activity, including plumes and cryovolcanism, persists on bodies that are small, cold, and distant from the Sun.

Standard formation models explain the initial distribution of volatiles through accretion histories and radial sorting in protoplanetary disks. What remains less clear is \textbf{why volatiles continue to be mobile, influential, and abundant within planetary interiors over geological timescales}. On a cooling, differentiating body, low-density, chemically active components are expected to migrate upward, react into stable phases, and gradually lose their capacity to drive interior dynamics. The persistence of volatile-rich interiors therefore raises questions that extend beyond initial delivery.

\section{Structural Consequences of Long-Term Interior Evolution}

If planetary interiors were passively cooling remnants of formation, their large-scale structure would be expected to stabilize early and evolve primarily through surface modification. Instead, several bodies exhibit macroscopic structural features that imply prolonged internal activity capable of reshaping planetary-scale geometry well after initial differentiation.

Earth’s unusually thick continental crust reflects sustained buoyant differentiation over billions of years, not a single early episode. Mars displays a pronounced hemispherical crustal dichotomy that is difficult to reconcile with simple cooling or impact histories alone, suggesting long-lived internal asymmetry. The divergent evolutionary paths of Earth and Venus—despite similar size and bulk composition—likewise point to differences in how internal processes operated over time, rather than differences confined to initial conditions.

These features are commonly attributed to early stochastic events or isolated mechanisms. However, their persistence and scale are consistent with interiors that remain dynamically active, continuously redistributing energy and material. In this sense, large-scale structure is not an independent anomaly, but a cumulative record of prolonged interior evolution—another indication that planetary bodies are not merely cooling systems, but continue to reorganize internally over geological timescales.

\section{A Missing Organizing Principle}

Individually, sustained internal heating, volatile persistence, excess planetary luminosity, subsurface oceans, and structural asymmetries can be accommodated within existing models. Collectively, they form a coherent pattern: \textbf{planetary interiors appear to undergo prolonged energetic and compositional evolution rather than simple monotonic cooling}.

What is missing is not observational data, but a unifying framework that treats these phenomena as related outcomes of common internal processes. Such a framework would need to account simultaneously for long-term internal energy, the persistence and influence of volatiles, and large-scale structural evolution across planetary bodies of very different types.

\section{Scope and Direction}

This note does not propose a specific mechanism, nor does it seek to reinterpret individual observations in detail. Its purpose is to motivate a shift in perspective—from viewing planetary interiors as largely exhausted systems toward viewing them as evolving systems with sustained internal drivers.

A separate work develops a coherent physical framework addressing these issues in a unified way, exploring implications for planetary interiors, volatile behavior, structural evolution, and broader astrophysical contexts. Readers interested in the proposed mechanism and its qualitative predictions are directed there \cite{tonon2mm}.

\printbibliography

\end{document}
