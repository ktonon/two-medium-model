\subsection{Gravity: Long-Range GCM Shadowing and the Energy Balance Problem}

In this framework, gravity arises from long-range shadowing of the Gravity-Carrying Medium (GCM). Localized concentrations of matter partially obstruct the background GCM flux, producing regions of reduced momentum intensity. When two masses are near one another, each resides within the other’s reduced-flux region and therefore experiences a slight imbalance in momentum transfer. The resulting net inward push is interpreted as gravitational attraction. On macroscopic scales, gravity reflects the cumulative effect of such shadowing interactions, structured by the configuration of the surrounding Light-Carrying Medium (LCM).

A longstanding difficulty for momentum-flux models of gravitation is the so-called overheating problem. If matter continuously intercepts a high-speed background flux, then the associated momentum transfer would be expected to deposit large amounts of energy, rapidly driving physical systems to unobserved temperatures. In naive formulations of Le Sage–type theories, this objection is severe, as no clear mechanism exists to prevent gravitational interactions from manifesting primarily as heating rather than as a stable long-range force.

Any theory based on gravitational shadowing must therefore confront this issue directly. The present framework does not claim an immediate or complete resolution of the overheating problem. However, it reframes the question by identifying additional irreversible channels beyond kinetic energy and thermalization. In particular, it allows for the conversion of gravitationally supplied energy into new rest mass under appropriate conditions, providing a potential sink that is absent from classical formulations. Whether this channel is sufficient, and how its contribution scales with object size, composition, and environment, remains an open question that requires further quantitative development.

A natural question arises here: are planetary interiors really dense enough to push LCM compression toward the thresholds required for wave confinement and matter creation?
This section only outlines the
conceptual mechanism. The next section,
\hyperref[earth-and-the-solar-system]{Earth and the Solar System}, examines
the planetary context more closely.
