\subsection{Gravity: Long-Range GCM Shadowing and the Resolution of the Heat Crisis}

In 2MM, gravity arises from \textbf{GCM shadowing}. Every standing
wave---proton, neutron, or electron---interacts inelastically with the
background GCM flux, reducing its intensity in the region directly
behind it. When two masses face one another, each sits inside the
other's reduced-flux region and therefore receives slightly more inward
momentum from the opposite side. The imbalance produces a \textbf{net inward push} that we interpret as gravitational attraction. On
macroscopic scales, what we call ``gravity'' is simply the accumulated
effect of countless such shadow interactions, structured by the LCM
fields of the matter involved.

The classical LeSage picture runs into trouble because it treats matter
as a rigid target being bombarded by a high-speed background, which
would quickly drive everything to unphysical temperatures. In 2MM,
particles are not solid absorbers; they are \textbf{standing waves} in
the LCM, confined by a balance between outward LCM elasticity and inward
GCM momentum. The GCM flux continuously supplies energy and momentum to
these structures. A significant fraction of that input is spent simply
on maintaining the compressed standing-wave configuration, while the
rest can appear as internal excitations or as kinetic energy of the
particle as a whole. In this sense, GCM interactions do not contribute to
kinetic particle energy (i.e., heating) to the extent one would expect
from their momentum transfer. The standing-wave structure provides a
robust channel for absorbing and redistributing energy rather than
allowing it to accumulate uncontrollably in purely translational motion.

At planetary scales, GCM interactions are therefore still expected to
contribute \emph{some} but not \emph{all} of their energy to internal
heating. In dense interiors, where LCM compression is high, wave
collapse and lag-mode collapse can occur, forming new protons and
electrons. The creation of new standing waves \textbf{consumes energy},
converting part of the gravitational power input into rest mass instead
of heat. Over geological timescales, this matter-creation channel acts
as an additional energy sink. 2MM does not claim that GCM-induced
heating disappears; rather, it proposes that some of that energy is
continually diverted into building and maintaining structure, including
the gradual addition of new matter in the deepest regions of massive
bodies.

A natural question arises here: \textbf{are planetary interiors really dense enough to push LCM compression toward the thresholds required for wave confinement and matter creation?}
This section only outlines the
conceptual mechanism. The next section,
\hyperref[earth-and-the-solar-system]{Earth and the Solar System}, examines
the planetary context more closely.

To conclude, gravity in 2MM is not a separate fundamental force but the
\textbf{macroscopic expression of GCM shadowing}, always shaped by the
LCM structures of matter. The heat crisis is avoided because both
particles and planets have built-in channels for GCM momentum to become
\textbf{organized structure}---stable standing waves and new
matter---rather than accumulating as unbounded thermal energy.
