\subsection{Nuclear Structure: Why Only Proton--Neutron Pairs Bind}

In 2MM, nuclear binding arises from the same interplay of LCM
compression fields and GCM shadowing that governs all other
interactions. What makes nuclei special is simply the \textbf{scale}:
particles must approach closely enough that their GCM shadows overlap
strongly while their LCM compression fields remain geometrically
compatible. Only one pairing---proton with neutron---achieves this
balance.

Protons resist close approach because their LCM compression fields are
extraordinarily steep and tightly wound. Pushing two protons together
forces their sharply peaked gradients to overlap, creating enormous
repulsive pressure long before their centers can approach the distance
required for strong GCM shadowing. Even though GCM provides an inward
momentum flux, the repulsion between two proton compression fields grows
much faster than the gravitational-like attraction. The result is
unavoidable: \textbf{proton--proton pairs cannot bind}, even if forced
together. Repulsion wins at every relevant scale.

Neutrons behave differently, but no more favorably for forming pairs.
Their compression fields are broader and softer, owing to their larger
and less compact standing-wave structure. This reduces both the
repulsion between two neutrons and the strength of their GCM shadowing.
Neither effect wins decisively. The repulsion is weaker than between
protons, but the attraction is weaker still, so the two neutrons can
never be brought close enough for the GCM flux to dominate. The outcome
is the same as for protons: \textbf{neutron--neutron pairs do not form stable bonds}.

The proton--neutron pair is the exception because their complementary
structures allow them to approach one another more closely and more
comfortably than like--like pairs. The proton's compact core can slip
deeper into the neutron's softer compression field without encountering
the catastrophic repulsion that would occur between two protons. This
closer separation creates dramatically stronger GCM shadowing between
their centers---far stronger than in either like--like case---because
shadow strength increases steeply with decreasing distance.

At the same time, the neutron's gentle compression gradient helps buffer
the proton's sharpest LCM features. The two fields can merge into a
configuration with \textbf{lower total deformation} than either particle
could achieve with a like partner. The neutron essentially absorbs some
of the proton's steep curvature while presenting little of its own. This
synergy allows the inward GCM momentum flux to outpace the rising LCM
repulsion, producing a genuine energy minimum---a bound nuclear state.

In this view, the familiar ``strong force'' does not require a separate
mechanism. It is simply the extreme end of the same balance between LCM
compression and GCM shadowing that governs all interactions, made
visible only when particles are pushed to distances where their
compression fields and shadows interact at their steepest gradients. A
proton--neutron pair is the only configuration that satisfies this
balance. All nuclear structure---from deuterium to the heavy
elements---emerges from combinations of these paired interactions and
their geometric constraints.
