\subsection{Magnetic Forces: Torsion, Motion, and Collective Alignment}

Magnetic behavior in 2MM follows directly from the \textbf{torsional component} of the electron's helical standing-wave structure. Even when
an electron is stationary, its dual-mode oscillations twist the
surrounding LCM into a subtle helical pattern. This built-in torsion
acts as an \textbf{intrinsic magnetic dipole}, a tiny directional
feature of the electron's compression field. Because the helical mode
has a handedness, the dipole has a preferred orientation, and nearby LCM
reacts differently depending on how the dipole is aligned.

When an electron moves through the LCM, the torsional field surrounding
it is carried along and becomes slightly skewed. The LCM does not flow
uniformly around the electron's helix: one side encounters greater
torsional resistance, while the other side relaxes. The GCM flux also
participates in this asymmetry, exchanging momentum differently across
the distorted compression field. Together, these effects produce a
\textbf{torque} that tends to align the electron's intrinsic dipole with
its motion---much like a spinning object aligning in a flowing fluid,
except that the mechanism here is mediated by LCM shear and GCM
shadowing rather than hydrodynamic drag. This alignment tendency is the
microscopic origin of magnetic behavior in moving charges.

When many electrons move together---as in an electric current---their
individual dipoles become biased toward a common orientation. Each
electron contributes a small skewing of the LCM torsional field, and
together these skewings add up to a coherent \textbf{shear pattern}
encircling the flow. The GCM flux reinforces this pattern by responding
to the collective distortion of the LCM, producing a stable macroscopic
field around the current. The familiar circular magnetic field around a
wire is therefore the large-scale imprint of countless aligned
microscopic helices, all interacting with the two media in the same
direction.

A different form of alignment appears in ferromagnetic materials. Here,
electrons do not need to move through the LCM to align their dipoles.
Instead, lattice geometry and energetic constraints encourage certain
orientations of the intrinsic helical mode. In regions where the lattice
allows these orientations to lock together, the torsional fields of many
electrons become mutually reinforcing. Their compression and shear
patterns combine into a persistent macroscopic field even in the absence
of current. Ferromagnetism, then, is simply the \textbf{static alignment} of many intrinsic dipoles, whereas current magnetism is the
\textbf{dynamic alignment} produced by motion.

\subsubsection{Helicity as a Natural Outcome of Wave Collapse}

It is worth stepping back to recall why the helical standing-wave
structure appeared in the first place. It was not assumed---it emerged
by imposing a small set of physical requirements on how a high-frequency
transverse wave must collapse when forming a pair of opposite particles.
The collapse must: \textbf{(1)} produce two structures with equal energy
but opposite chirality, \textbf{(2)} preserve the symmetry of the
original wave, \textbf{(3)} generate stable standing waves supported by
both the LCM and the GCM, and \textbf{(4)} provide a natural geometric
distinction between the two modes. A dual-oscillation helical structure
is the simplest configuration that satisfies all of these constraints.

What makes this especially compelling is that the structure is
\textbf{not ad hoc}---the universe already reveals the same geometry on
macroscopic scales. When electrical currents flow freely through a
plasma, unconstrained by solid conductors, they spontaneously organize
into \textbf{helical, filamentary structures}. This large-scale behavior
mirrors the microscopic torsional patterns predicted for individual
charge carriers. The fact that plasmas naturally adopt helical
geometries strengthens the case that helicity is not an arbitrary choice
in 2MM but a recurring, scale-independent feature of how the two media
prefer to arrange flowing energy.
