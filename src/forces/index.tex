\section{Interactions: How Forces Arise from the Two Media}\label{3-interactions-how-forces-arise-from-the-two-media}

With the structure of matter established, the next step is to understand
how matter \textbf{interacts}. In 2MM, there are no separate
``fundamental forces'' in the conventional sense. All
interactions---electrostatic, magnetic, nuclear, and
gravitational---arise from the same underlying mechanism: the way
\textbf{LCM compression fields} and \textbf{GCM momentum flux} respond
to the presence, motion, and geometry of standing-wave particles.

Every particle creates a characteristic distortion in the LCM, shaped by
its internal oscillation pattern and maintained by its GCM shadow. When
particles approach one another, these distortions overlap, reinforce, or
interfere in predictable ways. The GCM flux responds to these
distortions as well, altering the inward or outward momentum transfer
between the two media. What we interpret as ``forces'' are simply
\textbf{the adjustments required for the two media to maintain equilibrium while accommodating multiple standing waves in close proximity}.

This framework removes the need to postulate separate fundamental
interactions. Instead:

\begin{itemize}
\tightlist
\item
  \textbf{Electrostatic forces} arise from how LCM compression fields
  overlap or oppose each other.
\item
  \textbf{Magnetic forces} arise from interactions between torsional
  components of standing waves and the relative motion of their
  compression fields.
\item
  \textbf{Nuclear forces} emerge from extreme GCM shadowing and the
  resulting high-pressure compression wells at very short range.
\item
  \textbf{Gravity} is the macroscopic result of accumulated GCM
  shadowing from many particles acting in concert.
\end{itemize}

By tracing all interactions back to a single mechanism---the dynamic
cooperation between LCM elasticity and GCM flux---2MM provides a unified
physical origin for force behavior across all scales. In this sense, the
model does not contain multiple independent forces at all. \textbf{Every interaction reduces to a single kind of physical event: an inelastic momentum exchange between the two media.} What appear to us as distinct
forces are simply different geometric outcomes of how particles and
their compression fields shape, redirect, or obstruct the continual
movement of the GCM through the LCM.

\subsection{3.1 Electrostatic Forces: Compression Geometry and GCM Support}\label{31-electrostatic-forces-compression-geometry-and-gcm-support}

Electrostatic behavior emerges naturally once particles are understood
as localized distortions in the LCM. Each charged particle produces a
characteristic \textbf{LCM compression field}, shaped by the geometry of
its standing-wave mode and maintained by its GCM shadow. When two such
fields approach one another, they must fit together in a way that
preserves equilibrium in both media. The ease or difficulty of this fit
determines whether the interaction is repulsive or attractive.

\textbf{Like charges repel} because their compression fields have
similar shapes and orientations. Bringing them together requires forcing
two incompatible LCM geometries to overlap, which creates steep
gradients and increases local compression. The LCM responds by pushing
the particles apart---the same way two stiff springs resist being
pressed into one another. GCM effects play a secondary role here:
because each particle's shadow is shaped by its own compression field,
bringing like charges together does not create a deeper combined shadow.
Without the reinforcement that opposite charges enjoy, the GCM
contributes little inward pull, leaving the LCM's geometric
incompatibility to dominate and drive the particles apart.

\textbf{Opposite charges attract} for the complementary reason. Their
compression fields have opposite orientations, allowing them to
interlock smoothly. The combined field has lower overall distortion than
either field alone, creating a more stable configuration. The GCM flux
enhances this by forming a deeper, more coherent shadow around the pair,
increasing the inward momentum transfer and gently pulling them
together.

In 2MM, electrostatic forces are therefore not ``generated'' by the LCM
or the GCM in isolation. They arise from the \textbf{balance between LCM deformation and GCM pressure}, with compression-field geometry providing
the dominant mechanism and the GCM flux supplying the continual momentum
exchange that maintains the equilibrium.

It is also worth noting that, at these scales, the \textbf{local gravitational attraction} between two extremely dense standing
waves---though small on macroscopic scales---may be more significant
than traditionally assumed. The relative contribution of this
short-range GCM-driven attraction remains an open question and will
require a more detailed mathematical treatment. What matters for now is
that electrostatic forces need no separate postulate: they follow
directly from how the two media respond when compression fields overlap.

\subsection{3.2 Magnetic Forces: Torsion, Motion, and Collective Alignment}\label{32-magnetic-forces-torsion-motion-and-collective-alignment}

Magnetic behavior in 2MM follows directly from the \textbf{torsional component} of the electron's helical standing-wave structure. Even when
an electron is stationary, its dual-mode oscillations twist the
surrounding LCM into a subtle helical pattern. This built-in torsion
acts as an \textbf{intrinsic magnetic dipole}, a tiny directional
feature of the electron's compression field. Because the helical mode
has a handedness, the dipole has a preferred orientation, and nearby LCM
reacts differently depending on how the dipole is aligned.

When an electron moves through the LCM, the torsional field surrounding
it is carried along and becomes slightly skewed. The LCM does not flow
uniformly around the electron's helix: one side encounters greater
torsional resistance, while the other side relaxes. The GCM flux also
participates in this asymmetry, exchanging momentum differently across
the distorted compression field. Together, these effects produce a
\textbf{torque} that tends to align the electron's intrinsic dipole with
its motion---much like a spinning object aligning in a flowing fluid,
except that the mechanism here is mediated by LCM shear and GCM
shadowing rather than hydrodynamic drag. This alignment tendency is the
microscopic origin of magnetic behavior in moving charges.

When many electrons move together---as in an electric current---their
individual dipoles become biased toward a common orientation. Each
electron contributes a small skewing of the LCM torsional field, and
together these skewings add up to a coherent \textbf{shear pattern}
encircling the flow. The GCM flux reinforces this pattern by responding
to the collective distortion of the LCM, producing a stable macroscopic
field around the current. The familiar circular magnetic field around a
wire is therefore the large-scale imprint of countless aligned
microscopic helices, all interacting with the two media in the same
direction.

A different form of alignment appears in ferromagnetic materials. Here,
electrons do not need to move through the LCM to align their dipoles.
Instead, lattice geometry and energetic constraints encourage certain
orientations of the intrinsic helical mode. In regions where the lattice
allows these orientations to lock together, the torsional fields of many
electrons become mutually reinforcing. Their compression and shear
patterns combine into a persistent macroscopic field even in the absence
of current. Ferromagnetism, then, is simply the \textbf{static alignment} of many intrinsic dipoles, whereas current magnetism is the
\textbf{dynamic alignment} produced by motion.

\paragraph{Helicity as a Natural Outcome of Wave Collapse}\label{helicity-as-a-natural-outcome-of-wave-collapse}

It is worth stepping back to recall why the helical standing-wave
structure appeared in the first place. It was not assumed---it emerged
by imposing a small set of physical requirements on how a high-frequency
transverse wave must collapse when forming a pair of opposite particles.
The collapse must: \textbf{(1)} produce two structures with equal energy
but opposite chirality, \textbf{(2)} preserve the symmetry of the
original wave, \textbf{(3)} generate stable standing waves supported by
both the LCM and the GCM, and \textbf{(4)} provide a natural geometric
distinction between the two modes. A dual-oscillation helical structure
is the simplest configuration that satisfies all of these constraints.

What makes this especially compelling is that the structure is
\textbf{not ad hoc}---the universe already reveals the same geometry on
macroscopic scales. When electrical currents flow freely through a
plasma, unconstrained by solid conductors, they spontaneously organize
into \textbf{helical, filamentary structures}. This large-scale behavior
mirrors the microscopic torsional patterns predicted for individual
charge carriers. The fact that plasmas naturally adopt helical
geometries strengthens the case that helicity is not an arbitrary choice
in 2MM but a recurring, scale-independent feature of how the two media
prefer to arrange flowing energy.

\subsection{3.3 Nuclear Structure: Why Only Proton--Neutron Pairs Bind}\label{33-nuclear-structure-why-only-protonneutron-pairs-bind}

In 2MM, nuclear binding arises from the same interplay of LCM
compression fields and GCM shadowing that governs all other
interactions. What makes nuclei special is simply the \textbf{scale}:
particles must approach closely enough that their GCM shadows overlap
strongly while their LCM compression fields remain geometrically
compatible. Only one pairing---proton with neutron---achieves this
balance.

Protons resist close approach because their LCM compression fields are
extraordinarily steep and tightly wound. Pushing two protons together
forces their sharply peaked gradients to overlap, creating enormous
repulsive pressure long before their centers can approach the distance
required for strong GCM shadowing. Even though GCM provides an inward
momentum flux, the repulsion between two proton compression fields grows
much faster than the gravitational-like attraction. The result is
unavoidable: \textbf{proton--proton pairs cannot bind}, even if forced
together. Repulsion wins at every relevant scale.

Neutrons behave differently, but no more favorably for forming pairs.
Their compression fields are broader and softer, owing to their larger
and less compact standing-wave structure. This reduces both the
repulsion between two neutrons and the strength of their GCM shadowing.
Neither effect wins decisively. The repulsion is weaker than between
protons, but the attraction is weaker still, so the two neutrons can
never be brought close enough for the GCM flux to dominate. The outcome
is the same as for protons: \textbf{neutron--neutron pairs do not form stable bonds}.

The proton--neutron pair is the exception because their complementary
structures allow them to approach one another more closely and more
comfortably than like--like pairs. The proton's compact core can slip
deeper into the neutron's softer compression field without encountering
the catastrophic repulsion that would occur between two protons. This
closer separation creates dramatically stronger GCM shadowing between
their centers---far stronger than in either like--like case---because
shadow strength increases steeply with decreasing distance.

At the same time, the neutron's gentle compression gradient helps buffer
the proton's sharpest LCM features. The two fields can merge into a
configuration with \textbf{lower total deformation} than either particle
could achieve with a like partner. The neutron essentially absorbs some
of the proton's steep curvature while presenting little of its own. This
synergy allows the inward GCM momentum flux to outpace the rising LCM
repulsion, producing a genuine energy minimum---a bound nuclear state.

In this view, the familiar ``strong force'' does not require a separate
mechanism. It is simply the extreme end of the same balance between LCM
compression and GCM shadowing that governs all interactions, made
visible only when particles are pushed to distances where their
compression fields and shadows interact at their steepest gradients. A
proton--neutron pair is the only configuration that satisfies this
balance. All nuclear structure---from deuterium to the heavy
elements---emerges from combinations of these paired interactions and
their geometric constraints.

\subsection{3.4 Gravity: Long-Range GCM Shadowing and the Resolution of the Heat Crisis}\label{34-gravity-long-range-gcm-shadowing-and-the-resolution-of-the-heat-crisis}

In 2MM, gravity arises from \textbf{GCM shadowing}. Every standing
wave---proton, neutron, or electron---interacts inelastically with the
background GCM flux, reducing its intensity in the region directly
behind it. When two masses face one another, each sits inside the
other's reduced-flux region and therefore receives slightly more inward
momentum from the opposite side. The imbalance produces a \textbf{net inward push} that we interpret as gravitational attraction. On
macroscopic scales, what we call ``gravity'' is simply the accumulated
effect of countless such shadow interactions, structured by the LCM
fields of the matter involved.

The classical LeSage picture runs into trouble because it treats matter
as a rigid target being bombarded by a high-speed background, which
would quickly drive everything to unphysical temperatures. In 2MM,
particles are not solid absorbers; they are \textbf{standing waves} in
the LCM, confined by a balance between outward LCM elasticity and inward
GCM momentum. The GCM flux continuously supplies energy and momentum to
these structures. A significant fraction of that input is spent simply
on maintaining the compressed standing-wave configuration, while the
rest can appear as internal excitations or as kinetic energy of the
particle as a whole. In this sense, GCM interactions do contribute to
kinetic particle energy (i.e., heating) to the extent one would expect
from their momentum transfer. The standing-wave structure provides a
robust channel for absorbing and redistributing energy rather than
allowing it to accumulate uncontrollably in purely translational motion.

At planetary scales, GCM interactions are therefore still expected to
contribute \emph{some} but not \emph{all} of their energy to internal
heating. In dense interiors, where LCM compression is high, wave
collapse and lag-mode collapse can occur, forming new protons and
electrons. The creation of new standing waves \textbf{consumes energy},
converting part of the gravitational power input into rest mass instead
of heat. Over geological timescales, this matter-creation channel acts
as an additional energy sink. 2MM does not claim that GCM-induced
heating disappears; rather, it proposes that some of that energy is
continually diverted into building and maintaining structure, including
the gradual addition of new matter in the deepest regions of massive
bodies.

A natural question arises here: \textbf{are planetary interiors really dense enough to push LCM compression toward the thresholds required for wave collapse and matter creation?} Section 2 only outlines the
conceptual mechanism. The companion page
\href{https://github.com/ktonon/two-medium-model/blob/main/earth-and-solar-system.md}{Earth and the Solar System} examines
the planetary context more closely.

To conclude, gravity in 2MM is not a separate fundamental force but the
\textbf{macroscopic expression of GCM shadowing}, always shaped by the
LCM structures of matter. The heat crisis is avoided because both
particles and planets have built-in channels for GCM momentum to become
\textbf{organized structure}---stable standing waves and new
matter---rather than accumulating as unbounded thermal energy.

\begin{quote}
\textbf{Earth Expansion as a Research Context}

The idea that Earth's radius may have changed over geological time has
appeared in many forms throughout the history of geophysics. Although it
sits outside the modern mainstream, it has been examined seriously by
researchers such as Samuel Warren Carey, who viewed expansion as a
possible unifying framework for continental drift, and Dr. James Maxlow,
who produced detailed reconstructions suggesting smaller ancient Earth
radii. A full discussion of the evidence and counter-arguments is beyond
the scope of this paper; readers interested in the broader debate are
encouraged to consult those primary works.

What matters for 2MM is not the historical argument itself, but the
striking fact that the model approaches the possibility of planetary
growth from a completely different direction. Rather than beginning with
geological interpretation, 2MM arrives at the idea through the internal
physics of the LCM. Under sustained high compression, the model predicts
that LCM-wave collapse and proton formation act as a continuous energy
sink in planetary interiors, naturally introducing slow mass increase
over geological timescales.

One of the long-standing obstacles for expansion hypotheses has been the
absence of a physically grounded mechanism capable of driving sustained
planetary growth. 2MM does not claim to resolve the entire geological
debate, but it does offer a plausible mechanism where none previously
existed---providing a fresh context in which expansion models can be
reconsidered.

A more detailed discussion of how 2MM predicts Earth expansion can be
found at \href{https://github.com/ktonon/two-medium-model/blob/main/earth-and-solar-system.md}{Earth and the Solar System}.
\end{quote}
