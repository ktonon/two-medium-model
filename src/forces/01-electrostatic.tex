\subsection{Electrostatic Forces: Compression Geometry and GCM Support}

Electrostatic behavior emerges naturally once particles are understood
as localized distortions in the LCM. Each charged particle produces a
characteristic \textbf{LCM compression field}, shaped by the geometry of
its standing-wave mode and maintained by its GCM shadow. When two such
fields approach one another, they must fit together in a way that
preserves equilibrium in both media. The ease or difficulty of this fit
determines whether the interaction is repulsive or attractive.

\textbf{Like charges repel} because their compression fields have
similar shapes and orientations. Bringing them together requires forcing
two incompatible LCM geometries to overlap, which creates steep
gradients and increases local compression. The LCM responds by pushing
the particles apart---the same way two stiff springs resist being
pressed into one another. GCM effects play a secondary role here:
because each particle's shadow is shaped by its own compression field,
bringing like charges together does not create a deeper combined shadow.
Without the reinforcement that opposite charges enjoy, the GCM
contributes little inward pull, leaving the LCM's geometric
incompatibility to dominate and drive the particles apart.

\textbf{Opposite charges attract} for the complementary reason. Their
compression fields have opposite orientations, allowing them to
interlock smoothly. The combined field has lower overall distortion than
either field alone, creating a more stable configuration. The GCM flux
enhances this by forming a deeper, more coherent shadow around the pair,
increasing the inward momentum transfer and gently pulling them
together.

In 2MM, electrostatic forces are therefore not ``generated'' by the LCM
or the GCM in isolation. They arise from the \textbf{balance between LCM deformation and GCM pressure}, with compression-field geometry providing
the dominant mechanism and the GCM flux supplying the continual momentum
exchange that maintains the equilibrium.

It is also worth noting that, at these scales, the \textbf{local gravitational attraction} between two extremely dense standing
waves---though small on macroscopic scales---may be more significant
than traditionally assumed. The relative contribution of this
short-range GCM-driven attraction remains an open question and will
require a more detailed mathematical treatment. What matters for now is
that electrostatic forces need no separate postulate: they follow
directly from how the two media respond when compression fields overlap.
