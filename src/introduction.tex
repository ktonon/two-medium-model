\section{Introduction}

The Two-Medium Model (2MM) is a conceptual framework developed to
explore whether many pieces of modern physics might be understood more
intuitively if reinterpreted through two interacting physical media.
This includes particle behavior, forces, gravity, cosmology, and certain
geophysical observations. The goal of this work is not to dispute the
empirical success of established theories such as quantum field theory
or general relativity, but to examine whether a medium-based ontology
can offer a unified geometric picture that connects phenomena across
scales.

Modern physics is extraordinarily effective at describing what happens,
yet often agnostic about \emph{what things are}. Fields are treated as
mathematical entities, forces arise from operators or curvature rather
than physical mechanisms, and fundamental particles are defined by their
properties rather than by internal structure. These approaches work, but
they leave open the question of whether a more physically intuitive
model might also exist. Even better, if such a model did exist that it
might provide \textbf{new} insights.

The Two-Medium Model starts from a small set of philosophical
assumptions: that interactions are local, that waves require a medium,
and that gravity must be physically mediated. These assumptions are
neither new nor universally accepted, but they open a path to
constructing an alternative conceptual picture that remains grounded in
observed phenomena. Beginning from these premises, the model introduces
two distinct media: a compressible medium that supports transverse waves
(LCM) and a high-speed particulate flux that produces gravitational
shadowing (GCM). Matter, forces, and other phenomena are interpreted as
emerging from the interaction between these two components.

This document summarizes the resulting framework. It is not a formal
theory and does not attempt to replicate the mathematical precision of
modern physics. Instead, it offers a structured conceptual model that
appears internally coherent and that naturally produces explanations for
a range of phenomena, from electric charge and magnetism to nuclear
structure, planetary mass growth, redshift, and the function of active
galactic nuclei. Some of these explanations parallel mainstream
thinking; others are unconventional. All are presented with the
intention of sparking curiosity and inviting further analysis.

Each section of this paper should be read in order, the model benefits
from seeing how its pieces interlock: forces emerging from medium
interactions, particles from wave collapse, cosmology from matter
cycling, and planetary behavior from long-term LCM dynamics.

This work is offered as an exploration --- an attempt to see what
becomes possible when we start with different foundational assumptions
and follow them consistently. If it prompts reflection, critique, or
further development by readers more skilled or knowledgeable than
myself, then it has achieved its purpose.
