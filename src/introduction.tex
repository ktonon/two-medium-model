\section{Introduction}

This document explores whether a coherent, medium-based picture of physics can be constructed that connects phenomena across scales, from subatomic behavior to cosmology, using simple and physically intuitive principles. Rather than proposing a formal theory, it presents a conceptual framework intended to clarify mechanisms that are often treated abstractly in modern physics.

Contemporary physical theories are extraordinarily successful at prediction, yet they frequently remain noncommittal about ontology. Fields are defined mathematically, forces emerge from operators or curvature, and fundamental particles are characterized by properties rather than internal structure. While effective, this approach leaves open the question of whether a more explicitly physical, mechanism-based description might also exist—one that preserves empirical success while offering greater geometric and intuitive coherence.

The Two-Medium Model (2MM) begins from a small set of foundational assumptions: that interactions are local, that waves require a supporting medium, and that gravity must be physically mediated rather than action-at-a-distance. These assumptions are neither new nor universally accepted, but they provide a consistent starting point for constructing an alternative conceptual picture grounded in observed phenomena.

From these premises, the model introduces two interacting components: a compressible medium that supports wave motion (LCM), and a high-speed particulate flux that gives rise to gravitational effects through shadowing and pressure imbalance (GCM). Within this framework, matter, forces, and large-scale structures are interpreted as emergent outcomes of medium interactions, wave confinement, and energy exchange between these two components.

This paper summarizes the resulting framework in a structured, narrative form. It does not attempt to reproduce the mathematical precision of quantum field theory or general relativity, but instead aims to provide an internally coherent conceptual model that naturally connects a wide range of phenomena, including electric charge, magnetism, nuclear structure, planetary mass evolution, redshift, and the behavior of active galactic nuclei. Some interpretations align with conventional thinking, while others depart from it.

The sections that follow are intended to be read sequentially, as the model builds cumulatively: particles emerge from wave confinement, forces from medium interactions, cosmological behavior from matter and energy cycling, and planetary-scale phenomena from long-term medium dynamics. The goal is not to assert correctness, but to explore what becomes possible when a different set of foundational assumptions is adopted and followed consistently.
