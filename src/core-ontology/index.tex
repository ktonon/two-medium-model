\section{Core Ontology: Two Interacting Media}

The Two-Medium Model begins with a simple idea: the physical world
arises from the interaction of two different substrates. One medium
carries waves and can form standing-wave particles. The other supplies a
fast-moving background flux that shapes how those waves behave and how
particles influence one another. Most of the complexity of physics, in
this picture, comes not from many different forces, but from the way
these two media continually feed back on each other.

\subsection{The LCM: A Medium for Waves and Matter}

The Light-Carrying Medium (LCM) is a compressible elastic substance that supports compression, shear, and torsional distortions. Traveling waves with phase-locked shear and torsional oscillations correspond to what we ordinarily call light.

What makes the LCM especially important is how it stores and manipulates
energy. It can hold energy as density gradients, torsional motion, or
shear patterns, and under extreme conditions it can reorganize traveling
waves into localized standing structures. These standing waves become
what we recognize as matter. Because the LCM deforms around these
structures, it also shapes how the GCM flux interacts with them,
indirectly setting the stage for every familiar force.

The LCM does not independently ``produce'' electric or magnetic fields.
Instead, the deformations it carries---compression patterns, twists, and
gradients---provide half of the mechanism that later becomes
electrostatic, magnetic, and nuclear behavior. In this model, electric
and magnetic fields are simply the ways compressed or twisted regions of
the LCM guide the flow of the GCM.
 
\subsection{The GCM: A Fast Flux That Generates Forces}

The Gravity-Carrying Medium (GCM) is a very different kind of substrate.
It consists of an ultra-fast, high-flux stream of tiny particles moving
through space. When these particles pass through matter, a small
fraction of them interact in slightly inelastic ways. These tiny
encounters create ``shadows'' in the flux, which show up as the force we
call gravity.

The strength of a shadow depends on the structure of the LCM around the
particle. A tightly compressed standing wave blocks or redirects more of
the GCM than a loose one, so momentum flows inward more strongly. In
this way, the GCM does not just produce gravity---it participates in
everything from the cohesion of atomic nuclei to the subtle repulsive
and attractive forces that arise between charged particles.

Crucially, the GCM and LCM are deeply interdependent. The GCM shapes the
LCM by pushing on it, compressing it, or twisting it around standing
waves. And the LCM, in return, determines how much of the GCM penetrates
or is scattered by a region. Together, they create the entire landscape
of forces we observe. Later sections discuss how this interaction also
avoids the overheating problem that plagued earlier LeSage-type
theories.

In this ontology, the GCM provides the \textbf{momentum-transfer component}
of all physical interactions, while the LCM provides the
\textbf{geometric and structural component}. What we call ``forces''
emerge from the interplay between the two.

Although the individual ingredients of this framework---the idea of a
wave-supporting medium, a momentum-carrying flux, and standing waves as
particles---each have historical precedents, their combination into a
single interacting system is what gives 2MM its explanatory power. By
treating the LCM and GCM as mutually shaping components of one physical
ontology, the model brings together ideas that were previously separate
and shows how they can work together to produce the full range of
observed phenomena.

\subsection{Very Different Scales: Why Two Media Can Coexist}

A key feature of 2MM is that the two media do not operate at the same
scale. The LCM behaves like a smooth, continuous medium at the scales
relevant to light and matter. The GCM, by contrast, is made of extremely
small, extremely fast particles. This large separation in scale is what
allows both media to occupy the same space without simply behaving like
one blended substance.

In the spirit of Tom Van Flandern's meta-model, the GCM particles are
taken to be \textbf{far smaller than any length scale we can currently probe},
effectively much smaller than the Planck length for practical
purposes. Their role is not to form visible structure, but to provide a
nearly uniform, ultra-fine background flux that transfers momentum.
Because these particles are so small and so numerous, they can stream
through the LCM with almost no disturbance, except in regions where the
LCM is highly compressed or organized into standing waves.

The GCM flux is also assumed to move \textbf{much faster than light}.
This idea, again inspired by TVF's arguments for a superluminal gravity
medium, allows gravitational effects (in the shadowing sense) to
propagate effectively instantaneously at the scales we observe, without
conflicting with the observed behavior of light. In 2MM, light is
limited by the properties of the LCM, while the GCM operates on a
deeper, faster layer.

This separation of scale and speed has three important consequences:

\begin{itemize}
\tightlist
\item
  The \textbf{LCM} can carry waves and form standing-wave particles
  without being torn apart by the GCM.
\item
  The \textbf{GCM} can provide a persistent, high-speed momentum flux
  that responds sensitively to LCM compression and standing-wave
  structure.
\item
  Together, they can generate complex behaviors: gravity, fields, and
  forces; without requiring either medium alone to ``do everything.''
\end{itemize}

2MM builds on these scale assumptions and extends them: the GCM is not
only a gravity carrier, but also the mechanism that confines standing
waves, maintains particle compression, and couples to LCM deformations
to produce all familiar interactions. The extreme smallness and high
speed of the GCM particles are what make it possible for the two media
to coexist and yet play very different physical roles.
