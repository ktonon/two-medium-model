Up to this point, the Two-Medium Model has focused on waves, particles,
and forces. But these ideas rest on a broader cosmological backdrop that
now needs to be stated explicitly.

\textbf{2MM adopts, as an axiom, that the universe is infinite in extent, unbounded in scale, and without a beginning or an end.}

This is not presented as something proved by the model, nor as something
required by observation. It is simply the cosmological starting point
from which the rest of the theory is explored.

Given this assumption, one implication follows naturally:

If matter can be created in regions of high LCM compression, then
\textbf{other regions must exist where matter can be broken down}.
Otherwise, mass would accumulate indefinitely within an infinite,
timeless universe.

From this perspective, matter destruction is not a claim about what
\emph{must} happen in nature, but rather a logical requirement
\textbf{within the chosen cosmological framework} of 2MM. It becomes a
counterpart to matter creation, ensuring that the two-media system can
maintain long-term balance.
