\subsection{The CMB in an Infinite, Non-Expanding Universe}

In the Two-Medium Model, the universe is infinite and filled with countless sources of light: stars, galaxies, and all other standing-wave structures that emit radiation. But unlike in an expanding-space picture, light in 2MM does not stretch because space grows. Instead, every traveling wave gradually \textbf{loses a tiny amount of energy} to the GCM as it moves. Its frequency drops little by little, until eventually it falls below the threshold of what we can detect. In an infinite universe, this means that no matter how far you look, all of the distant light has already “run down” into extremely low frequencies.

When you add together the contribution from every direction: light from near sources, partially redshifted; light from further sources, heavily redshifted; and light from extremely distant sources, redshifted almost to nothing—you end up with a smooth background made of the \textbf{accumulated leftovers} of all waves that have lost most of their energy. Because this energy-loss process acts the same on all wavelengths and happens uniformly throughout space, the combined radiation naturally approaches a \textbf{blackbody spectrum}. A blackbody is simply the most “relaxed” or “smeared-out” form that radiation can take when its detailed structure has been erased, and that is exactly what the long-term redshifting process produces.

This idea solves Olbers' paradox. The night sky is not bright, even in an infinite universe, because every distant source has already faded far down the frequency spectrum by the time its light reaches us. Instead of a blinding sky, we see only the final, uniform glow of this accumulated low-frequency tail. In 2MM, this glow is the cosmic microwave background: the faint blackbody radiation produced by the infinite sum of redshifted waves from all directions, reflecting the steady running-down of light across the universe.
