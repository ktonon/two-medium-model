\subsection{Voids as Repulsive Regions}

In the usual cosmological picture, voids are simply large empty regions that expand faster than their surroundings. They do not actively push on matter. Some modified-gravity theories have suggested otherwise, proposing that voids might behave as if they were repulsive under certain conditions. But these models introduce new fields or modify gravity directly; none explain repulsion from the ground-up behavior of a medium. In contrast, the Two-Medium Model arrives at repulsive voids naturally by looking at how the two media exchange momentum.

The basic idea is simple. The GCM carries an extremely fast ``flux'' that pushes on matter. Whenever this flux passes through a dense region—like a galaxy, filament, or wall—it loses momentum because it interacts with tightly packed LCM structures there. Over time, dense regions drain and weaken the flux. But in a very empty region, like a cosmic void, the opposite happens. The GCM rarely interacts with anything and picks up a tiny amount of momentum from the loose LCM waves that fill the void. Because these waves nudge the GCM from all directions, sideways effects cancel out, but the small forward-moving component builds up. This means the GCM becomes more energetic the deeper it travels into a void.

It helps to visualize this as a kind of ``elevation map'' of flux strength. Voids sit at the \textbf{high-elevation} end of the map, because the GCM becomes most recharged there. Dense structures sit at the \textbf{low-elevation} end, because they drain the GCM. Matter always feels a push downhill on this map, away from high elevation and toward low elevation. Since voids are the high points, the push is always outward. This is the 2MM explanation for why voids act as repulsive regions: matter is simply sliding downhill in the landscape shaped by the GCM flux.

Real voids are not perfect spheres, so their ``elevation map'' is full of ridges, plateaus, and valleys. The highest points lie farthest from all surrounding structure. From these high points, many downhill paths lead toward the walls that border the void. Matter inside a void behaves like water on a tilted surface: it cannot climb toward the highest points and instead flows along the downhill channels. As these channels converge near the void walls, they produce small buildups of LCM density, which is where matter is most likely to form or grow. This explains why void centers remain nearly empty while galaxies cluster around the boundaries.
