\subsection{Quasars}\label{quasars}

Quasars present one of the more intriguing open questions for the Two-Medium Model. Empirical patterns noted by Halton Arp and others suggest that quasars often appear in structured pairs along the axes of active galactic nuclei, frequently aligned with observed jet directions. In several cases, the quasar pair shows a systematic trend in redshift: the quasar farther from the parent AGN tends to exhibit a slightly lower redshift than the one closer in. Although these associations remain debated within mainstream cosmology, the recurring geometric patterns—AGN core, bipolar alignment, paired quasars, and correlated redshifts—offer a valuable set of observational constraints.

Within 2MM, these clues may point toward a connection between matter destruction in AGN cores, the ejection of high-energy LCM wave structures along the jet axis, and the possibility that these structures evolve into quasar-like systems. Jet pinching near the AGN provides a natural location where intense energy outflows encounter temporary confinement, potentially shaping or stabilizing the initial form of these ejected structures before they propagate outward. The apparent redshift trend with distance could then reflect the changing balance of GCM flux as an ejected object moves from a heavily depleted region near the AGN, through localized pinched zones, and into progressively more recharged environments. However, this interpretation remains preliminary. The formation, evolution, and redshift behavior of quasars within 2MM have not yet been modeled in detail and remain an active area of exploration.

Quasars therefore represent both an opportunity and a challenge for the model: they exhibit large-scale patterns that seem naturally compatible with two-medium dynamics, yet require deeper theoretical development before firm conclusions can be drawn.
