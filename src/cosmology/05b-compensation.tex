\subsection{Compensation of LCM Deformation}

In developing the Two-Medium Model, we began with an assumption: the LCM is a continuous, nearly perfectly elastic medium. It supports compression, shear, and torsion, and it stores deformation as elastic strain rather than dissipating it through flow or permanent change. This assumption is motivated by the observed long-range, low-loss propagation of light, which strongly suggests an underlying medium with minimal internal friction.

Once this assumption is made, an immediate geometric problem appears. If standing-wave particles locally compress or “condense” the LCM, that deformation cannot exist in isolation. In an elastic continuum, every compression must be balanced by a compensating deformation elsewhere. The question is not whether compensation exists, but \textbf{how it is distributed}: sharply or gradually, locally or over large distances.

At the particle scale, early discussions of 2MM treated standing-wave particles as localized compression structures. This picture remains useful, but it implicitly neglected the compensating field. If compensation is gradual—as expected in a smooth elastic medium—then each particle must be surrounded by a broad, low-amplitude halo in which the LCM slowly relaxes back toward its ambient state. At large distances, this halo dominates; near the core, the nonlinear standing-wave structure dominates. In this sense, the original particle picture is a \textbf{near-field approximation} that remains valid when interactions are dominated by the core.

Clues about the nature of compensation appear at larger scales. Around planets and stars, the collective effect of vast numbers of standing waves produces a measurable condensation of the LCM, inferred from the bending of light as it passes nearby. This tells us that local compressions can add coherently and produce macroscopic effects. Crucially, we do not observe sharp shells of “missing” medium around these bodies, suggesting that the compensating deformation is not local or abrupt, but spread smoothly over large volumes.

When we zoom out further—to clusters, superclusters, and cosmic voids—the picture becomes clearer. Most of the universe’s volume contains very little matter. In 2MM, these voids have already been understood as regions needed to recharge and symmetrize GCM flux. They also naturally serve a second role: they provide the vast spatial reservoir required to accommodate the long-range compensating decompression of the LCM. Matter-rich regions become zones of net compression bias, while voids act as regions where the medium is correspondingly relaxed.

Taken together, this suggests a consistent, scale-spanning geometry. Standing-wave particles create localized compression cores with broad compensating halos. These halos overlap and sum at larger scales, producing measurable effects around stars and planets. At the largest scales, the compensation is exported into cosmic voids, whose immense size reflects the gradual, nonlocal nature of elastic balance in the LCM.

Under this view, earlier treatments of particles and interactions in 2MM remain valid as approximations—especially in the near and far field—but they sit within a deeper picture in which \textbf{no deformation is truly local}, and the structure of the universe itself participates in maintaining elastic balance.
