\subsection{Neutron Formation and Release in AGN Cores}\label{neutron-formation-and-release-in-agn-cores}

In 2MM, the stability of any particle depends on the cooperation of two factors: LCM compression, which creates a GCM shadow, and the availability of GCM flux, which reinforces that compression and keeps the standing wave confined. As matter is drawn toward the center of an active galactic nucleus, it encounters regions of steadily increasing LCM density. At the same time, the inward-moving GCM flux is gradually diminished by the surrounding material, so deeper layers receive progressively less of the confining pressure needed to sustain particle-scale structures.

In the outer portions of the core, the rising LCM density favors the formation of neutrons. Here, protons and electrons are driven into the paired lead–lag configuration that shares a common compression well. This arrangement becomes the most stable standing-wave structure under high compression, much as in neutron star matter. The GCM flux is still strong enough at these depths to maintain the neutron's compression pattern, allowing a neutron-rich layer to form.

Deeper in, however, the conditions change. Although the LCM becomes even more compressed, the GCM flux becomes too attenuated to support the neutron's shared compression well. Neutrons rely more heavily on ambient GCM reinforcement than protons do, and as the available flux drops below a critical threshold, their standing-wave structure can no longer remain confined. When this threshold is crossed, the neutron loses its internal compression support and ceases to exist as a localized particle. Its stored oscillatory energy is released back into the LCM as freely propagating traveling waves.

This transition marks the effective inner boundary of matter stability in an AGN core: outside it, neutrons form naturally under high compression; inside it, insufficient GCM support causes them to unravel and return their energy to the medium.
