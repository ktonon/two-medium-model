\subsection{Matter Destruction in Ultra-Dense Environments}\label{matter-destruction-in-ultra-dense-environments}

If the universe continually creates matter, a full equilibrium also requires a mechanism for matter destruction. In the Two-Medium Model, the most natural environments for this are the extreme-density regions associated with active galactic nuclei (AGN). In known astrophysics, neutron stars already demonstrate that protons and electrons can be forced into more compact states, forming neutrons to achieve tighter nuclear packing. If matter were to be broken down entirely, beginning with neutrons is the most plausible path, since their internal structure allows charge to be conserved when they are dismantled. This suggests that matter destruction would occur only in environments even denser than neutron stars—regions where all incoming matter is first compressed into neutron-rich form before any more fundamental breakdown can occur. For this reason, 2MM proposes AGN cores as the primary sites of matter destruction. While the process has not yet been modeled at the scale of the neutron, the working hypothesis is that neutrons entering these extreme environments are ultimately converted back into traveling LCM wave energy, completing the matter cycle required for equilibrium in an infinite universe.
