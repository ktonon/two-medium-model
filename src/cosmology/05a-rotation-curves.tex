\subsection{Galaxy Rotation Curves}

This same elevation landscape in the GCM of voids also explains why galaxies near voids often show rotation speeds that stay high far from their centers. In regions dominated by normal structure, the GCM flux is strongly depleted and behaves much like ordinary gravity, giving familiar Newton-like force laws. But on the side of a galaxy that faces a void, the galaxy encounters the highly recharged, partially isotropic flux coming from the empty region. This reduces the expected drop-off in gravitational push at large distances, flattening the rotation curve. No change to the laws of gravity is needed; the environment naturally reshapes the flux landscape.

Observed galaxy rotation curves, which remain high far from galactic centers, are usually explained by assuming large halos of invisible dark matter or by proposing that gravity itself changes at low accelerations, as in MOND. Both approaches fit the data, but neither is derived from basic physical processes. Dark matter is added as an unseen mass component, and MOND is introduced by modifying the force law. Neither provides a microscopic explanation for why galaxies should behave this way.

In summary, voids become repulsive ``high points'' in the GCM flux because empty regions recharge the flux while dense regions drain it. Matter moves downhill in this landscape, explaining why voids stay empty, why matter gathers at their walls, and why galaxies near voids exhibit MOND-like rotation speeds. All of this follows directly from the interaction rules of the two media, without introducing new fields or modifying gravity itself.
