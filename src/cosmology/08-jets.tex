\subsection{High-Energy Outflows and the Origin of AGN Jets}

If AGN cores act as matter-destruction engines, then the energy released from dismantling neutrons must reappear somewhere in the surrounding LCM. In 2MM, this energy re-emerges not as random radiation, but as highly organized, high-frequency LCM traveling waves. These waves carry away the liberated energy and momentum, and because the central region is extremely compact, the LCM flow naturally channels into narrow escape paths where the GCM–LCM environment offers the least resistance. This produces the familiar bipolar outflows known as AGN jets.

The collimation of these jets does not require magnetic fields or mechanical nozzles—though such effects may contribute in real astrophysical systems. In 2MM the jets arise simply because the LCM in the immediate polar directions is less dense and less turbulent than in the surrounding accretion structure. Once high-frequency LCM waves begin escaping along these paths, the geometry reinforces itself: outflow clears the channel further, allowing additional energy to escape the same way. The jets therefore form naturally as self-sustaining conduits for the release of matter-destruction energy.

From this perspective, AGN jets are the visible signature of the matter cycle closing: matter falls inward, is compressed to neutron form, is ultimately broken down into LCM traveling waves, and the released energy exits the system through highly collimated, relativistic outflows. This completes the balance required for a steady-state universe in 2MM.
