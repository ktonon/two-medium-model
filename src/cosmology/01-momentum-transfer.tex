\subsection{Momentum Flow}\label{momentum-flow}

We have already seen that the GCM couples strongly to regions of \textbf{high LCM compression}, transferring momentum into dense standing-wave structures and sustaining the confinement that defines matter. This one-way flow—from GCM into compressed LCM—is essential for particle stability. But on cosmic scales, a stable universe cannot operate as a pure sink: if the GCM continuously loses momentum to confined structures, it must also recover momentum somewhere else. The two media must exchange momentum in both directions to maintain long-term equilibrium.

This raises an important question: where does the GCM regain momentum?
Up to this point we have focused on compression, the mode that interacts strongly with the GCM. The other two deformation types—\textbf{shear and torsion}—were assumed to couple only weakly. But weak coupling does not mean negligible coupling, especially across the immense distances of intergalactic space. A tiny momentum exchange, applied repeatedly along long paths, can accumulate into a measurable effect.

This opens the possibility that \textbf{one or both of the low-compression modes—shear or torsion—may return momentum to the GCM rather than extracting it}. Unlike compression, which always slows the flux, shear and torsion do not significantly alter local LCM density. A GCM corpuscle passing through such a diffuse, low-impedance background may experience a slight net acceleration instead of a loss.

\textbf{Working assumption.} To make the dual-medium picture cosmologically viable, we assume that the GCM couples weakly but with opposite sign to one or both of the low-compression modes of the LCM (shear and/or torsion). Compression-rich regions always extract momentum from the GCM, while extended shear–torsion backgrounds return a small amount of momentum to the flux over large distances. The detailed microphysical origin of this sign reversal is left open, but it provides a minimal and physically plausible way for the two media to share a long-term momentum equilibrium.

\textbf{The implications of this assumption will be explored in the following sections, where it will be seen that the resulting large-scale behavior aligns cleanly with key cosmological observations.}
