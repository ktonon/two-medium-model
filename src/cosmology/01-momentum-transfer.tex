\subsection{Momentum Flow}\label{momentum-flow}

In the Two-Medium Model, gravity is produced by the flow of ultra-fast particles in the GCM pushing on patterns in the slower LCM. For this mechanism to work, the two media must exchange momentum in a way that depends on the local state of the LCM. Early versions of the model assumed that the momentum transfer switched direction depending on whether the region was a cluster or a void. That assumption turned out to be incorrect: galaxies exist inside voids, which means matter remains stable there, and so the rule could not depend purely on location. The model was updated accordingly.

The required sign flip is now tied to the \textbf{form of the LCM wave}, not to where the wave is located. The LCM supports two kinds of deformation: a \textbf{compression component}, which squeezes the medium, and a \textbf{torsion component}, which twists it. Light and matter both use these same two components, but in different proportions.

\textbf{Traveling waves} (light) need to move long distances without being scattered. They naturally evolve into a form dominated by \textbf{torsion}, which interacts only weakly with the GCM. When a GCM particle passes through this diffuse, torsion-heavy background, it receives tiny, random momentum kicks. Individually these kicks are small, but over long distances—such as the interior of a cosmic void—they add up to a slight \textbf{net gain of momentum for the GCM}. This is the recharge process that makes voids high-flux regions.

\textbf{Standing waves} (matter), on the other hand, need to stay in one place and resist spreading out. To do this, they must create a strong local disturbance in the LCM. This requires the compression component to dominate. A standing wave with strong compression presents a large “impedance” to the GCM: when a GCM particle strikes it, the GCM loses momentum efficiently. This loss is what creates the directional depletion, or “shadow,” that holds the standing wave together and makes matter stable.

With this picture in mind, the sign flip becomes natural:

\begin{itemize}
\tightlist
\item
  \textbf{Compression-dominated LCM (matter)} extracts momentum from the GCM.
\item
  \textbf{Torsion-dominated LCM (diffuse radiation)} gives a small net momentum back to the GCM.
\end{itemize}

This is a property of the \textbf{wave state} of the LCM, not the cosmic location. Void interiors happen to contain mostly diffuse, torsion-dominated LCM waves, so they are strong recharge zones for the GCM. Galaxies and other dense structures contain compression-dominated standing waves, so they always extract momentum from the GCM, even when placed inside a void.

This updated rule also resolves earlier contradictions. Since matter always extracts momentum from the GCM, it remains stable everywhere, including voids. And since voids predominantly contain torsion-type waves, they naturally elevate the GCM flux, producing the repulsive behavior that shapes the large-scale structure of the universe in 2MM.
