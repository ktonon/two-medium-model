\subsection{Momentum Flow in 2MM: Why Voids Flip the Sign}

The Two-Medium Model describes the universe as a dynamic balance between
two interacting substrates:

\begin{itemize}
\tightlist
\item
  the \textbf{GCM}, a sea of ultra-fast, ultra-small corpuscles,
\item
  and the \textbf{LCM}, an elastic medium that hosts matter, light, and
  all the structures we can observe.
\end{itemize}

Wherever these two media touch, they exchange momentum. Dense
regions---galaxies, stars, planets---absorb momentum from the GCM and
use it to maintain their internal compression. Light propagates through
the LCM, bending and shifting as the GCM flux shapes its path. And in
the vast intergalactic voids, where both media become extremely tenuous,
the nature of this exchange changes entirely.

This section explores how that change in momentum flow both solves the
cosmological redshift problem and closes the energy loop needed to keep
an infinite universe in equilibrium.

\subsubsection{Two Coupled Energy Reservoirs}

At its core, the GCM--LCM system behaves like two connected energy
reservoirs. Each one can be ``hotter'' or ``colder'' than the other, not
in the thermodynamic sense, but in terms of \textbf{how much momentum each medium carries relative to the equilibrium state of the environment.}

Whenever the balance tilts, momentum flows:

\begin{itemize}
\tightlist
\item
  from the \textbf{GCM into the LCM} when the flux is too strong for the
  local LCM structure,
\item
  and from the \textbf{LCM back into the GCM} when the flux is too weak.
\end{itemize}

The direction is not arbitrary---it is set by the environment. Dense
regions always favor the first case. Deep voids naturally fall into the
second.

\subsubsection{Dense LCM Regions: Where the GCM Loses Energy}

Inside galaxies, the LCM is highly compressed and structured. Atoms,
stars, and gas clouds are all standing-wave configurations that require
significant outward pressure from the GCM to maintain their shape. The
GCM corpuscles crash into these dense LCM structures, lose kinetic
energy, and carve shadows that show up macroscopically as gravity.

Here, the direction of momentum flow is unambiguous:

\begin{quote}
\textbf{GCM → LCM} is the dominant transfer in dense regions.
\end{quote}

These environments act as \textbf{GCM sinks}---draining flux energy to
support matter, maintain gravitational wells, and heat galactic
interiors.

\subsubsection{The Problem With an Infinite Universe}

If the GCM only ever loses energy in dense regions, then an infinite
universe cannot remain in equilibrium. Over long timescales, the flux
would cool, weaken, and eventually lose the ability to support matter
altogether. The universe would collapse or disperse, depending on how
the LCM responded.

But in 2MM we are assuming an \textbf{infinite, steady-state universe},
one where average properties---GCM intensity, LCM density, matter
formation rate---remain statistically constant over time.

For that to be possible, the direction of momentum flow must reverse
\textbf{somewhere}.

And there is only one natural place: the vast \textbf{intergalactic voids}.

\subsubsection{Voids: Where the Flux Runs Thin}

By the time GCM corpuscles leave a galaxy and drift into a void, they
have been weakened by countless inelastic collisions. Their flux is
real, but diminished---below the level needed to catalyze matter
formation or sustain tight LCM compression.

Meanwhile, the LCM in voids, though sparse, contains long-wavelength
structures such as \textbf{light} and residual background modes.
Compared to the starved GCM flux, these LCM excitations hold \emph{more organized energy} than the weakened flux can support.

This imbalance flips the sign of the momentum flow:

\begin{quote}
\textbf{In voids, the LCM gives momentum back to the GCM.}
\end{quote}

This is the missing recharge mechanism. Even in an infinite universe,
the GCM no longer decays away---it is continuously replenished wherever
the flux has fallen below equilibrium strength.

\subsubsection{The Role of Photons: A Gentle, Forward-Coherent Transfer}

Light plays an especially important role in this environment.

In voids, the LCM is extremely smooth and dilute. GCM corpuscles, being
minuscule compared to the LCM's effective units, do not collide with
discrete lumps the way they do inside galaxies. Instead, they nudge the
\textbf{bulk LCM field} in tiny, nearly elastic, highly forward-coherent
interactions.

From the photon's perspective:

\begin{itemize}
\tightlist
\item
  each interaction is incredibly weak,
\item
  happens over scales much smaller than its wavelength,
\item
  and changes its direction negligibly.
\end{itemize}

But over cosmological distances, even a tiny, uniform energy leak
becomes detectable:

\begin{quote}
\textbf{Photons lose a microscopic fraction of their energy into the GCM as they cross voids.} The result is a cumulative redshift.
\end{quote}

This accomplishes two things at once:

\begin{enumerate}
\def\labelenumi{\arabic{enumi}.}
\tightlist
\item
  \textbf{It recharges the GCM flux}, restoring it toward the
  equilibrium intensity needed to support matter elsewhere.
\item
  \textbf{It produces a cosmological redshift} without scattering or
  blurring distant galaxies---because the interactions are far too
  gentle to push photons off course.
\end{enumerate}

The redshift is not a Doppler effect, nor an expansion of space, nor a
magical ``tired light'' guess. It is the natural consequence of an
ultra-weak energy flow from LCM → GCM in zones where the flux has fallen
below equilibrium strength.

\subsubsection{Voids as the Universe's Reset Mechanisms}

Taken together, these insights reveal a satisfying, self-contained
cosmological loop:

\begin{itemize}
\tightlist
\item
  \textbf{Galaxies drain the GCM flux} (GCM → LCM).
\item
  \textbf{Voids restore the GCM flux} (LCM → GCM), primarily through the
  energy of long-traveling photons.
\item
  \textbf{Recharged flux reenters new regions}, regains its ability to
  bind matter, and supports the formation of future structures.
\item
  All of this occurs within a \textbf{time-independent, infinite   universe} in statistical equilibrium.
\end{itemize}

This explains why matter forms where it does, why voids remain empty,
and how redshift arises naturally without cosmic expansion.

Voids are not empty---they are \textbf{maintenance zones}, the regions
where the universe restores the strength of the GCM that galaxies have
drained.

\subsubsection{A Future Predictive Handle}

Because voids are where the flux recharges, and galaxies are where it
drains, the geometry of the universe---the spacing of galaxies, the size
of voids, the typical cluster density---may ultimately place
\textbf{constraints on the actual size, density, and cross-section of the GCM agents}.

If these agents were too large or too interactive, recharging would
overshoot equilibrium. If they were too small or too rare, the flux
would never fully recover. Matter stability, redshift gradients, and
void statistics all reflect these hidden parameters.

This opens the door for a richer, more quantitative model in the
future---one where cosmological structure becomes a direct probe of the
microscopic properties of the GCM.
