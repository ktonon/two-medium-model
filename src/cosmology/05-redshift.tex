\subsection{Cosmological Redshift}\label{cosmological-redshift}

In this picture, saying that the GCM ``gains momentum'' in diffuse regions is the same as saying that the LCM waves there slowly \textbf{lose} a tiny amount of their own energy. For traveling waves like light, this gradual loss appears as a slight reduction in frequency—a redshift. Over the enormous distances between galaxies, these tiny losses accumulate. This becomes the 2MM explanation for the cosmological redshift: light does not stretch because space expands, but because it gives up a very small amount of energy to the GCM as it travels.

A natural question is why this energy loss does \textbf{not} cause light to scatter or blur, since it is exchanging momentum with the GCM along the way. The reason is that the GCM particles are extremely small compared to the structure of the LCM wave. Each interaction is just a tiny “tap,” and most of the sideways taps cancel out because they come from all directions equally. For every small nudge to one side, there is nearly always another nudge in the opposite direction. The light wave therefore keeps its direction and sharpness over vast distances.
