\subsection{Jet Pinching and the Possibility of New Matter Formation}

Close to the AGN, the emerging jets must pass through regions of high LCM density and strong GCM depletion, producing natural bottlenecks where the outflow can temporarily narrow or “pinch.” In 2MM, this behavior follows from the interaction of the two media: the liberated LCM wave energy seeks the lowest-resistance channels, but the surrounding environment near the AGN is crowded with dense material and steep gradients in flux strength. The result is a constrained region where the jet briefly tightens before breaking through into lower-density surroundings. This is broadly consistent with observations of jet collimation zones in astrophysics, though the underlying mechanism differs.

These pinch regions may also represent zones where the conditions for new matter formation could be met. In the 2MM picture, matter creation occurs when high-frequency LCM waves become sufficiently confined to cast a growing shadow in the GCM and collapse into standing-wave structures. A jet pinch combines three relevant ingredients: unusually intense LCM wave energy, temporary confinement by local pressure gradients, and strong anisotropy in GCM exposure. Although highly speculative, this raises the possibility that new matter might occasionally form within or near these constricted regions. At present, this remains only a suggestion; exploring whether jet dynamics can trigger matter creation will require more detailed modeling and is left as future work.

