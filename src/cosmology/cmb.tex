\subsection{The Cosmic Microwave Background in an Infinite 2MM Universe}\label{the-cosmic-microwave-background-in-an-infinite-2mm-universe}

The Two-Medium Model begins from a bold but simple premise: the universe
is infinite, both in space and in time. No edge, no beginning, no
expanding fabric. Instead, everything we observe emerges from the
continual exchange of energy between two intertwined media---the
ultra-fast GCM corpuscles and the elastic LCM in which matter and light
reside.

The challenge, of course, is that the night sky is dark. An infinite
universe filled with luminous objects ought to blaze like a star, yet it
doesn't. And on top of that paradox sits the Cosmological Microwave
Background, a faint but perfectly uniform glow at 2.7 K that modern
cosmology treats as the afterglow of a primordial explosion.

The goal here is to show how the 2MM picture resolves these puzzles
naturally, without invoking a Big Bang or cosmic expansion. The answers
follow directly from the way light behaves as it travels through the
low-density LCM of intergalactic voids.

\subsubsection{The Starting Point: Olbers' Paradox in an Infinite Universe}\label{1-the-starting-point-olbers-paradox-in-an-infinite-universe}

Olbers' paradox is simple: If the universe is infinite and filled with
stars, then every line of sight should eventually encounter a star.
Adding up all those contributions should make the night sky as bright as
a stellar surface.

But it isn't. Why?

Traditional explanations invoke:

\begin{itemize}
\tightlist
\item
  a finite-age universe,
\item
  an expanding metric that redshifts distant light,
\item
  or a finite horizon set by cosmic history.
\end{itemize}

The 2MM rejects all three. So the paradox must be resolved in a way
consistent with:

\begin{itemize}
\tightlist
\item
  \textbf{an infinite universe},
\item
  \textbf{constant statistical equilibrium},
\item
  \textbf{cosmological redshift arising from LCM → GCM energy drift},
\item
  \textbf{no global evolution of average properties over time}.
\end{itemize}

The solution actually follows directly from the
\href{https://github.com/ktonon/two-medium-model/blob/main/cosmological-redshift.md}{redshift mechanism}.

\subsubsection{Light Traveling Through Voids: A Slow Energy Drift}\label{2-light-traveling-through-voids-a-slow-energy-drift}

In the 2MM, photons traveling through intergalactic voids pass through
extremely low-density LCM. In these regions, the GCM flux has been
partially depleted by its earlier interactions inside galaxies. The
local balance between the two media is shifted: the LCM (carrying
long-range waves like starlight) is effectively ``hotter'' than the
weakened GCM flux around it.

That imbalance drives a \textbf{tiny, forward-coherent transfer of momentum from LCM to GCM}. To the photon, this appears as a slight loss
of energy---a redshift that accumulates over vast distances.

Crucially:

\begin{itemize}
\tightlist
\item
  the interaction is extremely gentle,
\item
  it does not scatter the photon appreciably,
\item
  and it depends only on conditions in voids, not on cosmic age.
\end{itemize}

This is the 2MM explanation for cosmological redshift: \textbf{A photon gradually cools as it traverses low-density, weak-flux environments.}

\subsubsection{Applying This Mechanism to an Infinite Universe}\label{3-applying-this-mechanism-to-an-infinite-universe}

Now imagine all the starlight in an infinite cosmos subject to this
effect.

Every photon emitted from a distant star:

\begin{itemize}
\tightlist
\item
  travels through voids for millions or billions of light-years,
\item
  gives up a tiny fraction of energy over and over,
\item
  gradually shifts from visible to infrared,
\item
  then from infrared to microwave,
\item
  and finally into extremely low-energy radio waves.
\end{itemize}

Given enough distance---and in an infinite universe, there is always
enough distance---light from remote stars becomes so redshifted that it
no longer contributes to the visible sky at all.

This alone eliminates Olbers' paradox:

\begin{quote}
\textbf{The night sky is dark because almost all faraway light has been cooled into microwaves and longer wavelengths before reaching us.}
\end{quote}

No finite-age universe is required. No expanding metric is required.
Redshift alone does the job.

\subsubsection{Where Does the Lost Energy Go?}\label{4-where-does-the-lost-energy-go}

Energy cannot simply disappear. Every redshifted photon has given up its
energy to the GCM flux in voids.

Those voids are enormous, filled with GCM corpuscles continuously
absorbing:

\begin{itemize}
\tightlist
\item
  the faint, cooled remnants of starlight,
\item
  the low-frequency tail of galactic radiation,
\item
  and any long-wavelength modes that propagate without interacting
  strongly with matter.
\end{itemize}

Because light flows in from all directions, and because voids occupy
most of the universe's volume, the GCM in these regions absorbs a
tremendous amount of energy over time.

But equilibrium demands that this absorbed energy not accumulate
indefinitely. And indeed, the GCM continually redistributes it back into
the LCM through extremely gentle interactions.

The result is an ambient ``glow''---a stable background temperature that
reflects the balance between:

\begin{itemize}
\tightlist
\item
  energy drained from photons (LCM → GCM), and
\item
  energy redistributed back into the LCM (GCM → LCM).
\end{itemize}

This glow is what we observe as the cosmic microwave background.

\subsubsection{The CMB as the Equilibrium Temperature of Voids}\label{5-the-cmb-as-the-equilibrium-temperature-of-voids}

In the 2MM picture, the CMB is not a relic of a hot beginning. It is:

\begin{quote}
\textbf{the steady-state equilibrium between the slow cooling of light in voids and the continual recycling of that energy by the GCM.}
\end{quote}

The key features of the observed CMB fall naturally into place:

\paragraph{Uniformity}\label{uniformity}

Void conditions are extremely uniform across cosmic scales. The
background temperature reflects that uniformity.

\paragraph{Blackbody Spectrum}\label{blackbody-spectrum}

A medium that continually exchanges energy between two coupled fields
(GCM and LCM) inevitably reaches a thermal distribution. The resulting
equilibrium spectrum is a blackbody.

\paragraph{Temperature (\textasciitilde2.7 K)}\label{temperature-27-k}

This is simply the equilibrium point where:

\begin{itemize}
\tightlist
\item
  the energy lost by photons during cosmological redshift
\item
  is balanced by
\item
  the GCM's redistribution of energy back into the LCM.
\end{itemize}

The temperature does not encode the age of the universe; it encodes
\textbf{the balance point of an infinitely old system.}

\subsubsection{A Universe Maintained by Voids}\label{6-a-universe-maintained-by-voids}

In this view, voids serve a critical cosmological role:

\begin{itemize}
\tightlist
\item
  \textbf{Galaxies drain the GCM flux}, taking in energy to maintain
  their structure.
\item
  \textbf{Voids recharge the GCM flux}, receiving energy from passing
  photons.
\item
  \textbf{The balance of these two processes keeps the universe in   equilibrium.}
\end{itemize}

Thus the CMB is not a leftover from a fireball. It is the ongoing
whisper of an infinite universe maintaining its internal balance.

\subsubsection{The Clean Aesthetic of the 2MM Solution}\label{7-the-clean-aesthetic-of-the-2mm-solution}

What makes this explanation compelling is its simplicity:

\begin{itemize}
\tightlist
\item
  No beginning or end.
\item
  No global evolution of physics.
\item
  No cosmological expansion.
\item
  No mysterious inflationary era.
\item
  Just the natural outcome of two media interacting across vastly
  different density regimes.
\end{itemize}

Redshift is not a clue to a stretching metric. The CMB is not the
remnant of a cosmic explosion. Both are signatures of a universe that
has always been here, always in motion, sustained by the interplay
between GCM and LCM across galaxies and voids.
