\subsection{Maintaining Matter Balance in an Infinite Universe}

In the Two-Medium Model, the universe is taken to be infinite and in long-term equilibrium. For such a universe to remain stable, the two media must not only exchange energy smoothly but must also maintain a balance in the amount of matter present. The model provides a clear pathway for matter creation. A traveling LCM transverse wave can gradually gain energy through blueshifting as it enters increasingly dense LCM regions. As its frequency rises, its cross-sectional exposure to the GCM flux increases, allowing it to cast a growing shadow. This initiates a feedback loop in which the wave becomes increasingly confined. Once the confinement becomes strong enough, the wave collapses into a localized standing-wave pair, interpreted as an electron–positron creation event. In regions of very dense LCM—such as planetary or stellar interiors—the positron can undergo further confinement, eventually forming a proton.

This mechanism ensures a steady trickle of new matter throughout the universe. But in an infinite, statistically steady cosmos, continuous matter creation requires an accompanying process of matter destruction. Without such a process, matter would accumulate indefinitely, contradicting the assumption of long-term equilibrium. Identifying where and how matter is ultimately broken down is therefore essential to completing the model.
