\subsection{Quasars}\label{quasars}

In 2MM, regions of extremely high LCM compression can both
\textbf{destroy} matter (deep in AGN cores) and \textbf{create} new
matter (where high-frequency LCM waves collapse). If neutron destruction
releases intense bursts of high-frequency radiation, then wherever that
radiation encounters a \emph{local increase} in LCM density, wave
collapse can occur and new standing waves---electrons, protons, and
sometimes neutrons---can form.

The most natural astrophysical site for this is the \textbf{base of AGN jets}. Magnetic pinching, gravitational focusing, and collimation all
produce a temporary compression bottleneck. As outgoing radiation from
the AGN core passes through this region, it blueshifts into a denser LCM
environment where wave collapse thresholds can be exceeded. This
converts part of the outgoing energy into compact packets of newly
formed matter.

In the 2MM framework, these compact, high-energy newborn structures
correspond observationally to \textbf{quasars}. Their high redshift
follows naturally: quasars originate in unusually steep LCM compression
gradients, so the light they emit must climb out of a deeper
``compression well'' than light from their parent galaxy. As they travel
outward along the jet into regions of lower compression, their internal
standing-wave modes relax, and their observed redshift gradually
decreases---an effect long noted in quasar--AGN associations.

Thus, quasars appear in 2MM not as distant primordial objects, but as
\textbf{localized bursts of newly created matter} formed where AGN
outflows pass through temporary spikes in LCM density. This
interpretation provides a simple physical link between AGN activity,
matter recycling, and the unusually high redshifts associated with
quasars.
