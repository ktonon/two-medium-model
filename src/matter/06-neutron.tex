\subsection{The Neutron: A Paired Standing-Wave Structure}

With the proton and electron understood as distinct helical standing-wave modes stabilized by GCM confinement, the neutron can be viewed not as a fundamentally separate particle but as a paired configuration of these two modes. In the Two-Medium Model, a neutron forms when a lead-mode (electron-like) oscillation and a lag-mode (proton-precursor) oscillation become phase-locked within a shared compression well.

In this combined state, the opposite chiral tendencies of the two modes partially cancel, producing a geometry that is neither fully lead-mode nor fully lag-mode. The resulting LCM compression profile is gentler than that of a solitary proton: the concentrated lag-mode structure alone generates an extremely steep compression gradient, but the paired configuration spreads energy over a wider region. This yields a particle that interacts gravitationally like a proton yet lacks the sharp electrostatic signature of either charge.

Unlike the proton, whose extreme density allows it to remain compact in any environment, the neutron’s stability depends on its surroundings. Inside nuclei, overlapping GCM shadows and elevated LCM densities provide a supportive background that reinforces the shared compression well, allowing the paired modes to remain coherent. Outside the nucleus, ambient LCM density is too low to maintain the combined structure. The compression well weakens, the two modes separate, and the system relaxes into its components: a proton-like collapsed lag-mode, an electron-like lead-mode, and a brief outward fluctuation in the LCM—observed as a neutrino-like disturbance. No standalone decay particle is required; the “neutrino” is simply the outward redistribution of oscillatory energy released as the paired state breaks apart.

The neutron is heavier than a proton because it confines energy from both helical modes, yet it is less compact. The interference between their compression patterns limits how tightly either mode can draw in the surrounding LCM. The result is a particle with high internal energy but a shallower confinement profile—one that relies on nuclear environments for long-term stability. This reduced confinement plays a central role in nuclear structure, influencing why neutrons stabilize nuclei, how they arrange in characteristic ratios, and why only certain isotopes are viable.
