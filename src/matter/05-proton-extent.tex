\subsection{Spatial Extent of the Proton in the Two-Medium Model}

In 2MM, the proton is not a rigid, sharply bounded object. It is a localized standing-wave configuration of the LCM whose influence extends outward through a smooth gradient of compression. The formation of a proton begins with the collapse of a positron-like standing wave into a tighter, higher-frequency core. As this internal oscillation becomes more concentrated, it casts a stronger shadow in the GCM, increasing the inward-directed momentum transfer. This inward push compresses the surrounding LCM, forming an extended region where the medium is noticeably denser than its ambient state.

Crucially, this compressed region does not terminate at a well-defined radius. The GCM-induced confinement acts most strongly very close to the core, and its influence weakens gradually with distance. The result is a \textbf{continuous radial gradient}: high compression near the core, tapering smoothly outward until the background LCM density is reached. Because the confining pressure and the internal oscillation do not enforce any strict boundary, the proton’s “size” is only an effective one—the region within which scattering, charge distribution, or other interactions are significantly altered by this density gradient.

This picture naturally leads to a \textbf{fuzzy proton}, not a particle with a hard surface. The effective radius depends on the method and energy of probing, since high-energy interactions penetrate deeper into the compressed LCM surrounding the core, while lower-energy probes respond mainly to the outer portions of the gradient. This matches the observed behavior of protons in scattering experiments, where different measurement techniques infer slightly different radii, reflecting the underlying smoothness of the spatial profile.

Thus, in 2MM, the spatial extent of the proton arises from the \textbf{outward spread of compressed LCM around a tightly confined core standing wave}, producing a radially graded structure rather than a discrete boundary. The proton’s observable size is therefore an emergent property of the LCM–GCM interaction, not a sharply defined physical edge.
