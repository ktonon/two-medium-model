\subsection{The Particle Zoo}

In the Two-Medium Model, all particles arise from standing waves in the LCM, but not every wave pattern can become a stable particle. Two principles determine which structures persist:

\begin{itemize}
\tightlist
\item
\textbf{The GCM couples strongly to compression but weakly to shear and torsion.}
   Compression increases local LCM density and therefore produces a significant GCM shadow, generating the inward pressure needed for confinement. Shear and torsional oscillations, by themselves, remain largely transparent to the GCM and cannot produce meaningful inward flux.
\item
\textbf{Without compression, a wave cannot be trapped.}
   A standing wave composed only of shear and torsion interacts too weakly with the GCM to generate a confining shadow. Such a structure will not form a localized particle; it will propagate at or near the speed of light, dispersing once external conditions no longer compress it.
\end{itemize}

These principles imply that the variety of observed particles—the particle zoo—arises from different ways total energy can be distributed among the three oscillation modes. \textbf{Stable particles must contain enough compression energy to cast a persistent GCM shadow}, anchoring the shear and torsional components into a self-sustaining structure. Configurations with little or no compression form only transient states: short-lived resonances, mesons, or rapidly decaying excitations that disperse as soon as the confining environment weakens.

In this view, the particle zoo reflects a spectrum of possible standing-wave patterns, with stability determined primarily by how strongly a mode engages the compression channel and therefore the GCM confinement that gives particles their enduring form.
