\subsection{Why the Proton Remains Compressed in Low-Density LCM}

Once the lag-mode standing wave undergoes secondary collapse and
reorganizes into the compact helical structure we identify as the
proton, the new configuration becomes profoundly self-stabilizing.
Unlike the positron-like lag-mode at ordinary densities, the proton is
not a surface-level feature of the LCM---it is a deeply concentrated
energy structure whose stability arises from continuous interaction
between the two media.

During collapse, the proton develops:

\begin{itemize}
\tightlist
\item
  \textbf{extremely high internal energy density},
\item
  a \textbf{very strong GCM shadow}, and
\item
  a \textbf{steep LCM compression field} tightly wrapped around the
  standing wave.
\end{itemize}

These features do not depend on the external LCM environment. They arise
from the proton's \textbf{own geometry and GCM opacity}, which together
create a self-reinforcing pocket of compressed LCM. Even if the proton
moves from a dense LCM environment (where it formed) into a low-density
region (such as surface matter or vacuum), it does \textbf{not}
decompress.

The reason is that the GCM flux is pervasive and uninterrupted
everywhere:

\begin{itemize}
\tightlist
\item
  The proton's concentrated energy produces a large GCM shadow.
\item
  That shadow produces continuous inward momentum flux.
\item
  The inward flux maintains the LCM compression well around the proton.
\item
  The compression well locks the standing wave into its compact
  geometry.
\end{itemize}

Nothing about this mechanism relies on high ambient LCM density. Once
established, the proton's internal feedback loop keeps it stable:

\begin{enumerate}
\def\labelenumi{\arabic{enumi}.}
\tightlist
\item
  \textbf{High energy density → high GCM opacity}
\item
  \textbf{High opacity → strong inward GCM momentum flux}
\item
  \textbf{Inward flux → tightly compressed LCM}
\item
  \textbf{Tight compression → stability of the standing-wave geometry}
\end{enumerate}

To appreciate how dramatic this stability is, compare the proton's
internal density to ordinary matter. A typical rock has a density of
about 3,000 kg/m\(^3\), whereas the implied energy density inside a
proton is roughly 10\(^1\)\(^7\) kg/m\(^3\) --- fourteen orders of magnitude
higher. Even white dwarf material (\textasciitilde10\(^9\) kg/m\(^3\)) and neutron
star crusts (\textasciitilde10\(^1\)\(^4\) kg/m\(^3\)) fall far below this level. From
the proton's perspective, every macroscopic environment --- rock, air,
vacuum --- is effectively \emph{dilute}.

This also clarifies why gravity \emph{seems} negligible at atomic
scales. On macroscopic scales, a proton's total mass is tiny, so its
gravitational pull appears insignificant. But what matters in 2MM is not
total mass---it is \textbf{energy density}. The proton concentrates a
staggering amount of energy into an extremely small region, creating an
\textbf{intensely focused GCM shadow} right around itself. This produces
a strong local inward momentum flux, even though the long-range
gravitational effect is small. The proton's own geometry makes this
inward flux sharply confined and exceptionally strong. This locally
intense field is also what ultimately contributes to short-range nuclear
binding---the beginnings of the strong force---which will be developed
in a later section.

In this light, the proton's stability becomes intuitive. It is a tightly
bound, ultra-dense collapsed mode, sustained by continuous reinforcement
between the LCM compression field and the GCM momentum flux. The
everyday world simply does not have the capacity to disturb it.
