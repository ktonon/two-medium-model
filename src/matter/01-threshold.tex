\subsection{Frequency, Compression, and the Threshold for Collapse}

A key consequence of this picture is that the stability of light depends on its transparency to the GCM. At low frequencies, traveling shear–torsion waves are spatially diffuse and interact negligibly with the GCM flux; the elastic properties of the LCM restore their form as they move. But when such a wave enters a region of higher ambient LCM density—such as a gravitational well—it blueshifts: its frequency increases, its wavelength shortens, and the high-density crests of the shear motion pack more closely together. The wave occupies a smaller physical region and its energy density rises, making it a denser and more coherent target for the GCM.

This yields a simple rule:
\begin{quote}
\textbf{Blueshifting waves progressively increase their effective cross-section to the GCM flux.}
\end{quote}

Beyond a certain threshold, the wave is no longer transparent. The GCM begins scattering strongly enough to cast a genuine shadow. The resulting inward flux compresses the surrounding LCM, which tightens the wave further. A positive feedback loop forms: shorter wavelength → stronger shadow → stronger compression → still shorter wavelength. Once the compression passes a critical limit, the traveling wave can no longer sustain its extended structure, and collapse begins.

Regions of high ambient LCM density—near massive bodies or within planetary interiors—naturally drive this process by raising local wave frequency and GCM opacity. This is why electron–positron pair production is favored in such environments.

In this view, pair production is not a special process but the natural outcome of a traveling shear–torsion wave becoming sufficiently compact—and therefore sufficiently opaque to the GCM—that its only stable configuration is a pair of confined standing waves with complementary phase structures.
