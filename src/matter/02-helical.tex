\subsection{Dual-Mode Helical Pumps: The Structure of Matter}

The assumption of a push–pull asymmetry between the two standing-wave modes follows naturally from the way compression and shear interact in a nonlinear medium. In the LCM, regions of higher compression change the local wave speed, so the shear motion does not simply ride on top of the compression cycle—it is redirected and amplified differently depending on their relative phase. This kind of nonlinear coupling is well known in many physical systems: in acoustics, fluid membranes, and elastic solids \cite{10.1098/rstl.1884.0002, Riley2001, LandauElasticity}, a ±90° phase shift between expansion and shear produces opposite net flows even when the underlying motions are identical in energy. By analogy, a confined standing wave in the LCM can naturally adopt two stable phase-locked configurations: one where compression and shear reinforce outward motion of the surrounding medium, and another where they reinforce inward draw. These opposite dynamical tendencies are a familiar outcome of nonlinear wave–boundary interactions and provide a simple physical rationale for the electron–positron push/pull pair.

The torsional component of the wave does not participate in the push–pull asymmetry, but it plays a crucial geometric role in shaping the particle. As the standing wave oscillates, its torsional motion continually twists the local LCM around the wave’s axis, producing a stable helical distortion in the medium. This twist does not drive material outward or inward; instead, it organizes the shear and compression motions into a spiral pattern that repeats each cycle, allowing the structure to close on itself without drifting. In effect, torsion supplies the “handedness” and geometry of the particle’s form, imprinting a screw-like wrap on the surrounding LCM that remains locked in place as long as the standing wave persists.
