\subsection{Dual-Mode Helical Standing Waves: The Structure of Matter}

Once a high-frequency LCM wave becomes sufficiently opaque to the GCM,
the resulting shadow forces the wave to collapse into a tightly
localized configuration. But the collapse cannot produce an arbitrary
structure. It must preserve symmetry, conserve energy, and yield a
stable pattern that both media---the LCM and GCM---can support.

The only configurations that satisfy these conditions are
\textbf{standing waves composed of two coupled oscillation modes}:

\begin{enumerate}
\def\labelenumi{\arabic{enumi}.}
\tightlist
\item
  a \textbf{radial compression--expansion mode}, and
\item
  a \textbf{torsional shear mode}.
\end{enumerate}

These modes are natural to a compressible, elastic medium like the LCM.
However, in a freely propagating wave, they do \textbf{not} remain
locked in phase. Each mode travels according to its own restoring forces
and dispersive properties. The result is that, under normal conditions,
the radial and torsional components drift relative to one another---no
stable structure emerges.

During GCM-induced collapse, the situation changes completely. As the
traveling wave becomes opaque to the GCM, the shadow compresses the
surrounding LCM and confines the energy into a shrinking region. Under
this forced confinement, the two oscillations reorganize into a
configuration where they \textbf{become phase-linked}. Their relative
timing locks into a stable pattern, producing a \textbf{helical standing wave}.

This helical form is not arbitrary. It arises directly from combining a
radial pulse with a torsional twist in fixed phase relationship. And its
orientation depends entirely on which oscillation leads:

\begin{itemize}
\tightlist
\item
  If the torsional mode \textbf{leads} the radial mode, the result is
  the \textbf{lead-mode helix}.
\item
  If the torsional mode \textbf{lags} slightly, the result is the
  \textbf{lag-mode helix}.
\end{itemize}

These two helices are \textbf{mirror images}, equal in energy but
opposite in chirality. They naturally correspond to the two particles
produced in pair creation:

\begin{itemize}
\tightlist
\item
  the \textbf{lead-mode helix} → the electron
\item
  the \textbf{lag-mode helix} → the positron
\end{itemize}

Their opposite chirality accounts for their opposite electric charge,
while their identical energy follows from the symmetry of the collapse.
The helical geometry also explains familiar particle properties:

\begin{itemize}
\tightlist
\item
  \textbf{Electric charge} arises from the sign and orientation of the
  radial compression field.
\item
  \textbf{Magnetic dipoles} emerge from the torsional component.
\item
  \textbf{Spin and helicity} follow from the structural chirality of the
  standing wave.
\item
  \textbf{Stability} is maintained because the standing wave casts a
  strong GCM shadow, which continually reinforces the LCM compression
  well that confines it.
\end{itemize}

In this way, the dual-mode helical standing wave becomes the fundamental
unit of matter in 2MM. Electrons and positrons are not abstract point
particles but structured, self-sustaining oscillations, stabilized by
the feedback loop between the LCM and GCM. Their form is the direct
outcome of the symmetry and dynamics of collapse at the threshold where
a high-frequency wave becomes opaque to the GCM.
