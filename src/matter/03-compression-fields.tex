\subsection{The Origin and Motivation for LCM Compression Fields}

In the Two-Medium Model, a localized standing wave in the LCM is surrounded by a region of increased LCM compression. This “compression field” gives the particle a finite spatial extent and governs how it interacts with both the LCM and the GCM. The GCM shadow produced by a concentrated oscillation provides a natural source of inward momentum flux, but the \textit{shape} of the resulting compression field depends on how the standing wave itself couples to the medium.

In the previous section, we described particles as helical standing waves composed of two oscillatory components: a radial compression mode and a torsional mode. A central idea—introduced here—is that the \textbf{relative phase} of these two modes determines how the boundary of the standing wave exchanges momentum with the LCM. The goal is to explain why the lead-mode (electron) and lag-mode (positron) do not produce identical compression environments, but instead exhibit \textbf{opposite radial tendencies}:

\begin{itemize}
\tightlist
\item
  a \textbf{push-outward} bias for the lead mode,
\item
  and a \textbf{pull-inward} bias for the lag mode.
\end{itemize}

A simple linear coupling does \textbf{not} produce this behaviour. Therefore, the model makes the following explicit working assumption:

\textbf{Assumption:} \textit{The two phase-locked helical modes interact with the LCM in a nonlinear way that results in opposite radial biases—one tending to displace LCM outward, and the other tending to draw it inward. These opposite tendencies shape the particle's surrounding compression field.}

This assumption is motivated by these considerations:

\begin{itemize}
\tightlist
\item
  \textbf{Concentrated oscillations already generate inward confinement through the GCM shadow.}
   The collapse of a high-frequency LCM wave into a standing structure increases its opacity to the GCM flux. The resulting shadow produces a persistent inward momentum pressure. This effect provides the baseline confinement needed for any stable compression field.
\item
  \textbf{Opposite radial biases are a plausible outcome of nonlinear mode coupling.}
   The compression oscillation itself naturally introduces nonlinearity, because local LCM density—and therefore wave speed—varies over the cycle. In many physical systems with this kind of nonlinear coupling, two phase-locked modes can settle into distinct stable states with opposite radial tendencies. This makes an inward–outward asymmetry between the two helical modes a physically reasonable possibility, even though the detailed microphysical mechanism has not yet been derived.
\item
  \textbf{With this push–pull asymmetry in place, the model gains explanatory coherence across domains.}
   This inward–outward asymmetry is a cornerstone of the broader framework. Once assumed, it leads to a set of coherent predictions that align with observations across multiple domains—planetary structure, stellar environments, large-scale cosmology, and even the qualitative behaviour of the fundamental forces. The fact that these diverse implications emerge from a single underlying assumption is part of what motivates exploring the Two-Medium Model, even while the microphysical mechanism behind the asymmetry remains an open problem.
\end{itemize}

In summary, the existence of an inward- and outward-biased pair of helical modes is an \textbf{assumed but physically motivated feature} of the model. The GCM shadow provides the overall inward confinement; the asymmetric coupling between torsion and compression shapes how each mode modifies the surrounding LCM. A full dynamical derivation of this push–pull behaviour remains an open problem for future work.
