\subsection {The Origin and Motivation for LCM Compression Fields}

In the Two-Medium Model, every localized standing wave is surrounded by a region of elevated LCM compression. This ``compression field'' sets the particle's effective size and determines how it interacts with both media. The inward GCM flux created by the particle's shadow provides the baseline confinement, but the detailed shape of the compression field depends on how the standing wave's internal motions couple to the LCM itself.

A confined particle carries three oscillatory components: shear and torsion inherited from the original light wave, and a compression mode induced during collapse by the GCM–LCM interaction. The shear and torsion remain phase-locked, preserving the helical geometry of the structure. What distinguishes the two particle types formed in pair production is how the compression mode locks in phase with the inherited shear motion. A ±90° phase offset between these motions produces two opposite dynamical states:

\begin{itemize}
\tightlist
\item
the \textbf{lead-mode}, which tends to push the surrounding LCM outward (electron-like), and
\item
the \textbf{lag-mode}, which tends to draw it inward (positron-like).
\end{itemize}

A simple linear coupling cannot produce this asymmetry. The model therefore treats it as an explicit but physically motivated assumption:
\textbf{the compression–shear interaction in the LCM is nonlinear, and the two ±90° phase-locked states naturally settle into opposite radial behaviors.}

This assumption rests on three points. First, concentrated oscillations naturally generate inward confinement through the GCM shadow, providing the baseline conditions for a stable compression field. Second, nonlinear coupling between expansion and shear is well known to produce paired modes with opposite radial tendencies, making the push–pull duality physically plausible. Third, once this asymmetry is admitted, the broader framework becomes strikingly coherent: the same mechanism underlies qualitative predictions across planetary structure, stellar environments, cosmology, and even the behaviour of the fundamental forces.

In summary, the lead-mode and lag-mode represent two stable, opposite dynamical states of the same underlying three-component standing wave. The GCM shadow supplies the confining pressure, while the nonlinear compression–shear phase relation determines whether the surrounding LCM is drawn inward or pushed outward. A full dynamical derivation of this asymmetry is left as an open problem for future work, but the assumption is grounded in familiar physical behavior and provides the structural backbone of the model.
