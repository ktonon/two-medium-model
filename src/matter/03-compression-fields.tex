\subsection{The Origin of LCM Compression Fields}

When a traveling LCM wave collapses into a helical standing-wave
particle, its energy becomes highly concentrated. This concentration
dramatically increases the particle's \textbf{opacity to the GCM flux},
producing a strong and persistent GCM shadow. The shadow is not a
secondary effect---it is the mechanism that completes and stabilizes the
particle's formation.

The GCM shadow has two key consequences:

\begin{enumerate}
\def\labelenumi{\arabic{enumi}.}
\item
  \textbf{It pushes inward on the region surrounding the standing wave.}
  The reduced flux behind the particle creates a net inward momentum
  transfer from the GCM, constantly squeezing the LCM toward the center.
\item
  \textbf{It forces the LCM into a state of heightened compression.}
  This compression is not uniform but follows the geometry of the
  standing wave itself, imprinting the particle's oscillatory structure
  onto the surrounding medium.
\end{enumerate}

The result is a stable \textbf{LCM compression field} surrounding the
particle. This field is not an abstract ``charge distribution,'' nor is
it something imposed externally. It is simply the shape the LCM must
take to remain in equilibrium with the inward GCM momentum flux and the
internal structure of the standing wave.

Different helical modes produce different compression patterns:

\begin{itemize}
\tightlist
\item
  The \textbf{lead-mode helix} (electron) creates an outward-oriented
  compression gradient with one sign.
\item
  The \textbf{lag-mode helix} (positron) produces the opposite pattern.
\end{itemize}

Thus, \textbf{electric charge} emerges directly from the geometry of the
standing wave and the way it shapes LCM compression under GCM
confinement.

Likewise:

\begin{itemize}
\tightlist
\item
  the \textbf{torsional component} of the standing wave produces
  circulating shear in the LCM, giving rise to magnetic dipole moments,
\item
  while the combined compression-and-torsion field defines the
  particle's short-range interactions and its effective size.
\end{itemize}

Because the GCM continuously flows through everything, the GCM shadow
remains active at all times. It constantly reinforces the LCM
compression around the particle, preventing the structure from
dissipating. A particle is therefore not a static thing but a
\textbf{self-sustained dynamical object}:

\begin{quote}
a confined oscillation that remains stable only because the two
media---LCM and GCM---continuously maintain each other's structure.
\end{quote}

This explains why the compression field is always present:

\begin{itemize}
\tightlist
\item
  it does not arise spontaneously,
\item
  it does not depend on external charge carriers,
\item
  and it does not require postulating new forces.
\end{itemize}

It is simply the interplay of LCM elasticity and GCM momentum flux
working toward a stable equilibrium around a localized standing-wave
core.

This mechanism completes a conceptual chain:

\begin{quote}
\textbf{Concentrated wave energy → increased GCM opacity → persistent GCM shadow → inward flux → stable LCM compression field.}
\end{quote}

Matter's ``fields'' are therefore not separate entities but the natural
medium distortions required to confine a standing wave in a dual-medium
universe.
