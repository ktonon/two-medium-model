\begin{wrapfigure}{r}{0.58\textwidth}
\vspace{-\baselineskip}
\begin{framed}
\small
\noindent\textbf{Assumption: GCM--LCM Coupling}

In the Two-Medium Model, the interaction between the GCM and the LCM depends on the local mode of deformation and the degree of compression. The GCM couples strongly to regions of sustained LCM compression and only weakly to shear or torsional oscillations in low-density regions. The direction of momentum transfer is likewise mode-dependent: in compressed regions, momentum is transferred predominantly from the GCM to the LCM, whereas in uncompressed shear and torsional motions, momentum transfer occurs predominantly from the LCM to the GCM.

\medskip

The form of this coupling is not derived from a deeper physical theory but is adopted as a constitutive assumption. It is motivated by the fact that this single set of rules yields a coherent account of phenomena by enabling a stable energy balance between the two media, with consequences that extend naturally to large-scale cosmic structure and evolution and provide consistent explanations of observed behavior.
\end{framed}
\vspace{-\baselineskip}
\end{wrapfigure}

The LCM can deform in three independent ways: \textbf{compression}, which changes local density; \textbf{shear}, which displaces the medium laterally; and \textbf{torsion}, which twists it about a local axis. Traveling light consists of finite wave packets whose motion is carried by phase-locked shear and torsional oscillations. Although such waves do not contain a sustained compression component, the shear oscillation produces brief, localized increases in density at points of maximum displacement.

Under ordinary conditions, these transient density spikes are both weak and widely spaced. As a result, they interact only negligibly with the GCM flux, and traveling shear–torsion waves remain effectively transparent to it. The elastic response of the LCM restores the wave as it propagates, allowing light to travel freely through a uniformly dense medium without confinement or collapse.

The behavior of light therefore depends not only on its internal structure, but on how that structure interacts with the surrounding state of the LCM. Changes in ambient LCM density alter the spatial and temporal arrangement of shear-induced density fluctuations within a traveling wave, setting the stage for qualitatively different behavior under appropriate conditions.