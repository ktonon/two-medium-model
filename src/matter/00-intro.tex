Many earlier physical models have attempted to describe matter as some
form of localized wave. Classical ether theories imagined particles as
knots or disturbances in a single universal medium. More recent
``elastic aether'' approaches propose that mass, charge, and quantum
behavior arise from deformations or resonances in an elastic or fluid
vacuum. Soliton-based models likewise treat matter as stable wave
packets in nonlinear fields. These ideas share a common intuition:
matter may be an organized pattern of motion rather than a separate
substance.

What remains unresolved in all single-medium approaches, however, is the
\textbf{problem of confinement}. In a uniform elastic medium, waves
disperse; they do not naturally remain localized. A stable,
particle-like standing wave requires some form of boundary condition or
trapping mechanism, but a single-medium ether does not provide a clear
physical origin for such boundaries. This limitation has historically
constrained the explanatory reach of wave-as-matter theories.

2MM addresses the problem of standing-wave confinement by introducing a
second substrate, the GCM, whose ultra-fast particulate flux interacts
with the LCM. When transverse LCM waves concentrate into a small region, 
they locally reduce the incoming GCM flux, creating a gravitational shadow.
This shadow increases the surrounding LCM compression, and the increased
compression further confines the wave. The feedback continues until the
system reaches a stable equilibrium. The wave becomes
\textbf{trapped in a gravity-induced compression well of its own making},
forming a localized, self-sustaining standing wave.
This cannot be overstated: 
\textbf{a separate substrate is required to stabilize standing waves in the LCM}.
2MM addresses the problem of standing-wave confinement by introducing a second substrate, the GCM, whose ultra-fast particulate flux interacts with the LCM. When transverse LCM waves concentrate into a small region, they locally reduce the incoming GCM flux, creating a gravitational shadow. This shadow increases the surrounding LCM compression, and the increased compression further confines the wave. The feedback continues until the system reaches a stable equilibrium. The wave becomes trapped in a gravity-induced compression well of its own making, forming a localized, self-sustaining standing wave. This cannot be overstated: a separate substrate is required to stabilize standing waves in the LCM.

This dual-medium interaction supplies a natural mechanism for the
existence of stable matter waves---something difficult to obtain in any
single-medium framework. To my knowledge, no prior model combines a
compressible light-carrying medium with a separate gravitational flux in
such a way that \textbf{confinement, particle structure, and gravitational response emerge from the same underlying dynamics}.

The ability of 2MM to explain how standing waves become self-confined is
one of the model's strongest conceptual features. It allows matter to
arise not as an arbitrary postulate, but as a natural consequence of how
two complementary media shape and reinforce one another.

\begin{quote}
\emph{A particularly influential example for this work is \textbf{Tom Van Flandern's Meta Model}, which argued that the vacuum may consist of multiple interacting substrates and that gravity and light need not propagate through the same physical medium. While the Meta Model differs substantially from 2MM in mechanism and structure, its core idea, that distinct media may underlie different physical phenomena, provided the inspiration for 2MM.}
\end{quote}
