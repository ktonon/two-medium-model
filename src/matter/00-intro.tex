The LCM can deform in three independent ways: \textbf{compression}, which changes local density; \textbf{shear}, which displaces the medium sideways; and \textbf{torsion}, which twists it around a local axis. A traveling light wave consists solely of phase-locked \textbf{shear and torsional} motions. Although it carries no sustained compression component, the shear oscillation produces brief, localized increases in density at the points of maximum displacement.

A key assumption of 2MM is that the \textbf{GCM interacts strongly only with compression}, and only very weakly with shear or torsion. Momentary density spikes generated by shear are normally too small and too widely spaced to cast a meaningful shadow in the GCM flux, so ordinary light propagates freely through space.

As a traveling light wave enters a region of higher ambient LCM density, its frequency increases while its wavelength decreases, because the wave cycles faster through a medium that resists displacement more strongly. This shortening of the shear–torsion wavelength brings the brief high-density crests closer together. Even though each crest is only momentarily denser than the background, decreasing their spacing increases their \textbf{cumulative} opacity to the GCM. Once this combined shadow becomes large enough, inward GCM pressure begins reinforcing the densest points of the wave.

A positive feedback loop follows: tighter spacing → stronger shadow → stronger compression → even tighter spacing. The traveling wave collapses into a confined standing structure, acquiring a third oscillation mode—a \textbf{sustained compression component}—maintained jointly by inward GCM flux and the elastic response of the LCM.
