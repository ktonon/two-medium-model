\subsection{Frequency, Compression, and the Threshold for Collapse}

A key consequence of this picture is that the stability of light depends
on its transparency to the GCM. At low frequencies, LCM waves are
diffuse in absolute space and interact negligibly with the GCM flux.
They travel freely, their structure restored by the elastic properties
of the LCM.

As frequency increases, however, the wave's spatial extent shrinks
(again, in the absolute frame defined by the GCM). The crests bunch
closer together, the oscillation occupies a smaller physical region, and
energy density rises. From the GCM perspective, the wave becomes a
denser and more coherent target.

This leads to a simple but profound rule:

\begin{quote}
\textbf{High-frequency waves progressively increase their effective cross-section to the GCM flux.}
\end{quote}

At some point, the wave ceases to be transparent. The GCM begins
scattering off it strongly enough to cast a real shadow. The shadow
compresses the nearby LCM, which in turn constricts the wave. If the
compression crosses a critical threshold, the traveling wave loses the
ability to maintain its extended form.

This is the onset of collapse.

The earlier intuition---that a blueshifted or high-frequency wave
becomes unstable---is still partly correct, but the true driver is the
\textbf{interaction with the GCM}, not the elasticity of the LCM itself.
Regions with high LCM density (such as near massive bodies or inside
planets) naturally compress incoming waves, increasing both their local
frequency and their opacity to the GCM. This explains why pair
production is favored in such environments:

\begin{itemize}
\tightlist
\item
  \textbf{High local LCM density → higher local optical frequency}
\item
  \textbf{Higher optical frequency → increased GCM cross-section}
\item
  \textbf{Increased GCM opacity → stronger shadowing}
\item
  \textbf{Shadowing → collapse into standing waves}
\end{itemize}

Thus pair production is a natural consequence of the dual-medium
interaction. It occurs when a traveling wave becomes sufficiently
compact---and therefore sufficiently opaque to the GCM---that the only
stable configuration remaining is a pair of standing-wave particles with
complementary phase structures.

Here is the \textbf{fully updated Section 2.3}, rewritten from start to
finish to incorporate the improved wording and ensure a smooth, coherent
narrative consistent with your revised GCM-driven collapse model.

You can paste this version directly into your document.
