\subsection{The Proton: Why the Lag-Mode Collapses in Dense LCM}

The electron and positron arise when a high-frequency LCM wave becomes
opaque to the GCM and collapses into two helical standing-wave modes.
But under certain extreme conditions---particularly in regions where the
\textbf{ambient LCM density is already high}---the lag-mode structure
behaves in a qualitatively different way. It does not merely stabilize
as a positron. Instead, it undergoes a \textbf{secondary collapse} into
a much more compact standing wave: the proton.

The key lies in how the two chiral modes respond to rising LCM
compression.

The \textbf{lead-mode} (electron-like) standing wave resists
compression. Its phase alignment tends to push outward, thickening the
surrounding LCM compression field in a way that counteracts further
inward collapse. Even in dense environments, the lead-mode maintains its
characteristic size and structure.

The \textbf{lag-mode}, however, responds in the opposite way. Its phase
alignment draws LCM inward, amplifying the local compression rather than
resisting it. Under ordinary surface-of-Earth conditions, this inward
pull is far too weak to overcome the ambient LCM pressure; the
surrounding medium simply ``pushes back,'' stabilizing the lag-mode as
the familiar positron.

But in environments where the LCM is \textbf{already highly compressed}---deep inside planets, in dense plasma regions, or wherever
steep LCM gradients exist---the lag-mode's inward pull becomes
effective. The ambient medium no longer resists; it \emph{assists} the
contraction. The standing wave therefore begins to tighten, its helical
radius shrinks, and its GCM opacity rises dramatically.

A runaway process begins:

\begin{enumerate}
\def\labelenumi{\arabic{enumi}.}
\tightlist
\item
  \textbf{Contraction increases the GCM shadow.}
\item
  \textbf{A stronger shadow increases inward GCM pressure.}
\item
  \textbf{Inward pressure tightens the standing wave even further.}
\end{enumerate}

The structure collapses into a new equilibrium---one that is far more
compact, with vastly greater energy density, and a much deeper GCM
shadow. This new configuration is the \textbf{proton}.

The proton therefore is not just a ``heavier version'' of the positron.
It is the \textbf{collapsed-state form} of the lag-mode standing wave
under conditions where external LCM compression allows the inward phase
dynamics to succeed. Its key properties follow naturally:

\begin{itemize}
\tightlist
\item
  a much smaller radius
\item
  a deeper and more intense LCM compression field
\item
  vastly higher GCM opacity
\item
  far greater stability
\item
  the same positive charge as the positron (same chirality, different
  scale)
\end{itemize}

This explains why protons are ubiquitous in environments where pair
production is possible but LCM density is high, such as inside planetary
interiors. The environment selects the mode: the lead-mode stabilizes as
the electron, while the lag-mode collapses into the proton.

The proton's stability is not a coincidence; it is the endpoint of a
feedback loop between the internal standing-wave geometry and the
surrounding media. Once formed, the proton's extreme GCM opacity locks
it into a self-sustaining compression well, creating the remarkably
robust particle at the heart of ordinary matter.

\paragraph{Result: Hydrogen Formation}

In environments where the LCM is driven into unusually high
compression---such as planetary interiors or other steep-gradient
regions---the internal modes of matter respond in predictable ways.
Under these conditions, the lag-mode naturally contracts into a
proton-like state, while the lead-mode remains electron-like, creating a
straightforward pathway for hydrogen to form as a stable product of
extreme compression.

This mechanism preserves charge balance and operates independently of
any primordial supply of material. A discussion of why this remains
feasible even when naïve density estimates might suggest otherwise, and
how it shapes the evolution of planets and moons, is presented in 
\hyperref[earth-and-the-solar-system]{Earth and the Solar System}.
