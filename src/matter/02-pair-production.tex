\subsection{Criteria of Pair Production}

Pair production already tells us that matter does not emerge in isolation: when a high-energy light wave collapses, it produces two particles with exactly the same energy but opposite character. Whatever structure underlies these particles must therefore come in complementary forms that cannot be rotated into each other. In other words, the particles must represent two distinct chiral configurations of the same underlying standing wave. This requirement is not optional—it is built into the observed symmetries of pair creation. A viable model must therefore identify a single physical template that naturally admits two non-superimposable opposites with identical energy content but reversed dynamical tendencies. The phase structure of the three-mode standing wave provides the simplest route to this kind of duality.

In pair production, the collapsing light wave brings only two kinds of motion with it—shear and torsion. These are intrinsic to light itself, and when the wave becomes trapped they are reasonably assumed to keep the same phase relationship they had while propagating. The third component, compression, does not come from the light wave at all; it emerges only during confinement, as the inward GCM flux encounters the LCM’s resistance to being compressed. The key difference between the two particles produced in the collapse lies in how this newly generated compression mode locks in phase with the inherited shear–torsion pattern. A ±90° offset produces two opposite dynamical behaviors: one phase choice pushes the surrounding medium outward, while the opposite phase draws it inward. These two variants have identical energy content but reversed tendencies, providing exactly the kind of complementary pair demanded by pair production.
