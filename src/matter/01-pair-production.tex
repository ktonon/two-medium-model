\subsection{Pair Production as a GCM-Induced Collapse}

Pair production has a striking feature: a single high-energy photon
transforms into two particles---an electron and a positron---that are
equal in energy but opposite in structure. This symmetry suggests that
matter must arise from a wave configuration that can collapse into
\textbf{two complementary standing-wave modes}, differing only in their
internal phase alignment.

\begin{quote}
\textbf{Note on Terminology}

In this document, the word \emph{collapse} refers strictly to a
\textbf{physical contraction of an LCM wave}, a dynamical process in
which a high-frequency traveling wave becomes unstable and reorganizes
into a confined standing-wave structure. This is a mechanical
transformation of real oscillations in a medium. \textbf{It has nothing to do with quantum-mechanical wave-function collapse}, which is a
separate and unrelated concept. To avoid confusion, \emph{we will never use the term ``collapse'' to mean wave-function collapse} anywhere in
this work.
\end{quote}

In earlier drafts, the collapse mechanism was described as the LCM
``failing to support'' extremely high-frequency oscillations. Upon
closer examination, a more coherent explanation emerges once the GCM is
included. In 2MM, the trigger for pair production is not an elastic
limit of the LCM at all---it is the \textbf{GCM reacting to the growing opacity of the wave}.

From the GCM's perspective, a high-frequency LCM wave becomes
increasingly compact in \emph{absolute} space (the GCM reference frame).
As the energy is squeezed into a smaller region, the wave begins to
interact more strongly with the GCM flux. Once the wave's energy density
crosses a threshold, it becomes \textbf{partially opaque} to the GCM.
This opacity creates a small but rapidly intensifying \textbf{GCM shadow}.

The shadow compresses the surrounding LCM. The wave is no longer free to
propagate: the GCM, responding to the opacity, forces the oscillation
inward. What follows is a rapid collapse of the traveling wave into a
tightly localized configuration.

The collapse cannot simply extinguish the energy. The system must
preserve overall symmetry, geometry, and conservation laws. The natural
outcome is for the collapsing wave to reorganize into \textbf{two standing waves}, identical in energy but opposite in internal phase
ordering. These are the \textbf{lead-mode} and \textbf{lag-mode} helices
that form the electron and the positron.

Under this view, pair production is driven not by the LCM failing, but
by the \textbf{GCM enforcing a new equilibrium} when a traveling wave
becomes too compact to remain transparent. Matter arises from the
interaction between the two media at a sharp threshold of opacity.
