\section{Methodology and Acknowledgements}

The development of this model was highly iterative and heavily assisted
by modern AI tools. Without such tools, the model described here would
likely have taken months or years to assemble. As someone with a job, a
family, and limited hours to devote to speculative research, I would not
have been able to sustain the rapid cycle of hypothesis, critique, and
refinement that this project required. AI made that pace possible.

However, the role of AI requires careful qualification. These tools did
not ``invent'' the ideas presented here. Instead, they made the
collective knowledge of the scientific community navigable in a way that
was never previously accessible to individuals outside formal research
settings. During development, I would pose challenges and suggests
constraints, and the AI would respond with ideas, critiques,
counterexamples, or alternative possibilities drawn from patterns across
the literature it had been trained on. Most of these suggestions were
rejected. Only a small subset survived repeated rounds of filtering and
conceptual testing on my part.

A clear example of this collaborative dynamic is the dual-oscillation
model of the electron and positron. The idea of combining two
oscillatory modes with a phase offset was surfaced in dialogue with an
AI system. I would not have discovered that configuration unaided. But
the structural constraints that shaped it --- the insistence on mirror
symmetry, three-dimensional non-flippability, pair-production balance,
and geometric stability --- came from my own criteria. The AI generated
raw candidate ideas; I applied the conceptual standards that determined
whether they were tenable. The final standing-wave geometry emerged
through that interaction.

This dynamic applies throughout the project. The synthesis is my own.
But several individual components surfaced only because AI made it
possible to explore a wide conceptual space quickly. For that reason, I
want to acknowledge not only the role of AI but also the deeper lineage
behind it: the thousands of scientists, educators, writers, engineers,
and students whose work forms the substrate upon which such tools are
built. There is no practical way to identify every individual influence,
but their collective contribution underlies every AI-assisted iteration.

If you believe that any ideas in this document overlap with work you
have published, and you would like your contributions acknowledged or
referenced, please feel free to open an issue or submit a pull request.
I will review it and update the references accordingly.

Several strands of prior work influenced the direction of this model.
Halton Arp's empirical studies of redshift anomalies suggested that
large-scale cosmology might still contain unaddressed puzzles. Tom Van
Flandern's ``Meta-Model'' introduced the idea of multiple interacting
media, a conceptual seed that eventually grew into the Two-Medium
framework presented here \cite{MetaResearchStructureOfMatter2003}. Dr. Chantal Roth's modeling of an elastic
aether encouraged me to consider that empty space could contain more
structure than I had previously assumed. The collection of papers in
Pushing Gravity: New Perspectives on Le Sage's Theory of Gravitation,
edited and contributed to by researcher Matthew R. Edwards, introduced
me to Le Sage-type gravity models and the challenges they
entail---particularly the overheating problem. Matthew himself first
introduced me to Expansion Tectonics decades ago, when I was completing
my computer science major at the University of Toronto.

While these works are not mainstream, they demonstrated that alternative
ontologies can still be logically organized and empirically motivated.

I present this work not as a finished theory, but as an invitation: to
reconsider whether some aspects of physical law might gain clarity from
a different underlying ontology, and to explore whether the ideas
outlined here might be refined, challenged, or developed further by
those with greater expertise and technical resources.
