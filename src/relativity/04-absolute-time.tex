\subsection{Absolute Time, the GCM, and Faster-Than-Light Transmission}\label{absolute-time}

In 2MM, the presence of the GCM introduces a deeper notion of
\textbf{absolute time}---a timebase that exists independently of the LCM
processes that define our clocks and our experience. All measurement,
sensation, and physical activity in living systems occurs through
LCM-standing-wave structures, so we have access only to \emph{LCM time}.
But the GCM operates on its own timescale. It is not constrained by the
wave-propagation speed of the LCM (the speed of light), nor by the
relativistic limits that emerge from LCM-based processes.

This idea aligns with Tom Van Flandern's careful analysis in
\textbf{``\href{https://www.metaresearch.org/cosmology/cosmology2/the-speed-of-gravity-what-the-experiments-say}{The Speed of Gravity: What the Experiments Say}'' (1998)}, where he reviewed
timing delays, planetary motions, and signal propagation. His conclusion
was that gravity-like influences must propagate extraordinarily fast:

\begin{quote}
\textbf{Gravity must exceed 2 x 10\(^10\) c}, at least twenty billion times
the speed of light.
\end{quote}

In the context of 2MM, this is exactly the behavior expected of the GCM
flux. It moves on an inaccessible absolute timebase and is free to
propagate vastly faster than light. This immediately suggests that
\textbf{faster-than-light transmission} is physically possible if a
civilization can learn to encode patterns into the GCM.

But this does \textbf{not} permit faster-than-light communication in the
usual sense. The distinction is fundamental:

\begin{itemize}
\tightlist
\item
  \textbf{Transmission} is the propagation of patterns through the GCM.
\item
  \textbf{Communication} requires encoding and decoding, both of which
  happen in the LCM.
\end{itemize}

Encoding information into the GCM, and reading it out on the other end,
requires ordinary physical processes---electronics, detectors,
biological cognition---all of which are constrained by LCM time and the
speed of light. So even if a signal propagates nearly instantaneously
through the GCM, the total round-trip communication time cannot exceed
relativistic limits.

Thus:

\begin{quote}
\textbf{FTL transmission is possible in theory. FTL round-trip communication is not.}
\end{quote}

This preserves causality. No paradoxes arise because observers made of
LCM-standing-wave matter can never process a complete exchange faster
than light allows.

Yet the implications are profound. A civilization could, in principle,
carry on a \textbf{real-time conversation between Earth and Proxima Centauri}, because the GCM would carry the signal almost instantly. The
subjective experience would feel simultaneous, even though the
underlying encoding and decoding steps remain light-speed bound.

\paragraph{Implication for SETI}

This also leads to a natural expectation: if technological civilizations
develop a way to use the GCM for signaling, then \textbf{electromagnetic communication becomes obsolete}. EM transmission is slow, lossy, and
limited by LCM constraints, whereas the GCM allows nearly instantaneous
signaling over interstellar distances.

Thus, 2MM implies that \textbf{SETI's strategy of searching the electromagnetic spectrum may never detect advanced civilizations}, not
because they do not exist, but because their communication technology
would likely move beyond EM channels entirely. They would use the
GCM---an invisible, ultra-fast substrate that our current instruments
cannot yet manipulate or detect.

In this light, SETI's silence may be a clue rather than a contradiction:
advanced civilizations might be speaking in a medium we are not yet
listening to.
