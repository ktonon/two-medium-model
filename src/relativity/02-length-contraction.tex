\subsection{Why Motion Through the LCM Causes Length Contraction}

When relativity is taught using spacetime diagrams, length contraction
is often presented as a geometric effect: moving objects ``shrink''
along their direction of travel because of how coordinates transform.
While mathematically precise, that explanation can feel abstract and
disconnected from physical intuition.

In the 2MM picture, length contraction emerges far more naturally. It is
simply what happens when an object moves rapidly through a \textbf{real, compressible medium}: the LCM.

As an object accelerates, it must push its way through the LCM. At low
speeds this has almost no noticeable effect, but as the speed approaches
the propagation speed of transverse waves in the LCM (what we call the
speed of light), something important happens: the LCM begins to
\textbf{pile up} in front of the object.

This buildup creates a \textbf{directional compression gradient}:

\begin{itemize}
\tightlist
\item
  \textbf{Higher LCM density ahead of the object}
\item
  \textbf{Normal LCM density behind it}
\end{itemize}

This gradient grows steeper the faster the object moves. And because all
matter in 2MM consists of standing waves stabilized by the surrounding
LCM, a denser medium in front exerts a real, physical inward pressure on
the wave structure.

Crucially, there is \textbf{no pulling from behind}. The LCM behind the
object is simply less compressed; it does not contribute actively to the
contraction. Instead, it fails to counterbalance the inward pressure
from the front. The net effect is a \textbf{one-sided squeezing force}
along the direction of motion.

From an outside observer's perspective, the object is genuinely
shortened --- its entire structure is compressed along the direction of
travel by the piling up of LCM ahead of it.

But from the inside, nothing appears distorted.

All internal rulers, atoms, and even the biochemical processes that
constitute perception are made of the same standing waves immersed in
the same LCM. When the medium compresses in one direction, \textbf{every component of the object compresses together}. No internal measurement
reveals the contraction because the measuring instruments themselves
contract by the same proportion.

This perspective makes length contraction feel intuitive:

\begin{quote}
\textbf{Length contraction happens because fast motion through the LCM creates a directional compression gradient, and all matter --- being made of LCM standing waves --- must adapt to the compressed medium.}
\end{quote}

There is no paradox, no mystical geometric effect, and no need to
imagine ``shrinking'' as an illusion. It is the straightforward
mechanical response of waves interacting with a real medium.
