\subsection{Mutual Causation in the Two Media}

Any elastic-aether model naturally interprets spacetime curvature as
variations in the density or tension of the medium itself. What 2MM adds
is a dynamic role for gravity in \emph{maintaining} that curvature. In
the geometric view of relativity, the sequence seems straightforward:
matter and energy bend spacetime, and gravity is the motion of objects
through that curvature. In 2MM the picture becomes more intertwined. A
standing wave \emph{is} a region of denser LCM, yet that denser region
can only remain stable because the GCM flux continually reinforces it
through shadowing. At the same time, the GCM would have nothing to
shadow if not for the density created by the standing wave. The result
is a self-sustaining feedback loop between the two media---neither the
LCM nor GCM ``comes first,'' and no initial moment is required. The
system simply persists, with curvature, matter, and gravity all emerging
from their mutual interaction.

This interdependence between the two media leads to an intriguing
implication: for 2MM to function as described, the system cannot easily
accommodate a true ``beginning.'' A beginning would require one medium
to exist and operate without the other, or for standing waves to form
before the shadowing effects that sustain them --- both of which break
the causal loops that the model depends on. In fact, any notion of an
absolute beginning already strains causality in conventional physics as
well. In 2MM, this tension resolves naturally: the universe is not
something that starts and then evolves; it is something that
\emph{self-maintains}, with structure emerging continually from the
ongoing reciprocity between the LCM and GCM.
