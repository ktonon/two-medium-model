\subsection{Space-Time is Real}

In the 2MM framework a key insight is that everything normally
attributed to ``spacetime'' can be understood in much simpler, more
concrete terms if \textbf{the LCM itself \emph{is} what we experience as
space and time.} Not metaphorically. Literally.

The LCM is a physical, elastic medium. It fills all of space. It can be
compressed, stretched, and shaped by energy and matter. And we --- made
of standing waves in that medium --- are part of it.

Under this interpretation, relativity becomes natural rather than
abstract:

\begin{itemize}
\item
  \textbf{Mass doesn't ``curve spacetime.''} Mass simply compresses the
  LCM. Those compression gradients \emph{are} the curvature.
\item
  \textbf{Gravity isn't a mysterious bending of geometry.} It is the
  result of the GCM flux pushing on regions where the LCM has been
  compressed by standing-wave particles. The momentum flow follows the
  gradient of LCM density --- which is exactly what Einstein's field
  equations describe, just in a different language.
\item
  \textbf{Time dilation isn't some strange slowing of clocks.} Time
  slows down because \textbf{light waves take longer to propagate   through compressed LCM}, and every process in our bodies depends on
  those waves. If the medium compresses, the speed of light \emph{in   that medium} decreases --- and because we are made of the medium's
  standing waves, \emph{we slow down with it.} From the inside,
  everything still seems normal, because our measuring devices and
  biological processes slow in perfect sync with the light.
\item
  \textbf{The speed of light appears constant because we can't step   outside the medium to compare.} We only ever measure light using tools
  made of the same LCM. If the medium thickens or thins, our clocks
  thicken or thin with it. We never notice the change.
\end{itemize}

This resolves one of relativity's most confusing aspects: Why does the
speed of light seem constant from all perspectives, even when intuition
says it shouldn't be?

In 2MM, the answer is simple:

\begin{quote}
\textbf{We are creatures of the LCM. Everything we can measure, know, or experience is bound to the properties of this medium. We cannot perceive the LCM directly, because doing so would be like trying to look at our own eyes without a mirror.}
\end{quote}

From this viewpoint, relativity does not describe an abstract
four-dimensional manifold. It describes the behavior of a real physical
substrate under compression. Einstein's equations remain correct in
their predictions, but 2MM provides a different ontological picture:
instead of bending ``spacetime,'' nature is reshaping the LCM, and we
move, age, and measure within that reshaped medium.

This makes relativity not an exotic distortion of geometry, but a simple
consequence of the physics of waves in a compressible medium --- the
same medium that forms light, matter, and ourselves.
