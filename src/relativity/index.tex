\section{Relativity}

Relativity is often introduced through the language of \emph{spacetime curvature}---a mathematical surface that bends and twists in response to
mass and energy. For many people, this picture feels abstract and
untethered from anything physical. ``Curved spacetime'' is something you
can calculate, but not something you can visualize or touch.

\subsection{5.1 Space-Time is Real}

In the 2MM framework a key insight is that everything normally
attributed to ``spacetime'' can be understood in much simpler, more
concrete terms if \textbf{the LCM itself \emph{is} what we experience as
space and time.} Not metaphorically. Literally.

The LCM is a physical, elastic medium. It fills all of space. It can be
compressed, stretched, and shaped by energy and matter. And we --- made
of standing waves in that medium --- are part of it.

Under this interpretation, relativity becomes natural rather than
abstract:

\begin{itemize}
\item
  \textbf{Mass doesn't ``curve spacetime.''} Mass simply compresses the
  LCM. Those compression gradients \emph{are} the curvature.
\item
  \textbf{Gravity isn't a mysterious bending of geometry.} It is the
  result of the GCM flux pushing on regions where the LCM has been
  compressed by standing-wave particles. The momentum flow follows the
  gradient of LCM density --- which is exactly what Einstein's field
  equations describe, just in a different language.
\item
  \textbf{Time dilation isn't some strange slowing of clocks.} Time
  slows down because \textbf{light waves take longer to propagate   through compressed LCM}, and every process in our bodies depends on
  those waves. If the medium compresses, the speed of light \emph{in   that medium} decreases --- and because we are made of the medium's
  standing waves, \emph{we slow down with it.} From the inside,
  everything still seems normal, because our measuring devices and
  biological processes slow in perfect sync with the light.
\item
  \textbf{The speed of light appears constant because we can't step   outside the medium to compare.} We only ever measure light using tools
  made of the same LCM. If the medium thickens or thins, our clocks
  thicken or thin with it. We never notice the change.
\end{itemize}

This resolves one of relativity's most confusing aspects: Why does the
speed of light seem constant from all perspectives, even when intuition
says it shouldn't be?

In 2MM, the answer is simple:

\begin{quote}
\textbf{We are creatures of the LCM. Everything we can measure, know, or experience is bound to the properties of this medium. We cannot perceive the LCM directly, because doing so would be like trying to look at our own eyes without a mirror.}
\end{quote}

From this viewpoint, relativity does not describe an abstract
four-dimensional manifold. It describes the behavior of a real physical
substrate under compression. Einstein's equations remain correct in
their predictions, but 2MM provides a different ontological picture:
instead of bending ``spacetime,'' nature is reshaping the LCM, and we
move, age, and measure within that reshaped medium.

This makes relativity not an exotic distortion of geometry, but a simple
consequence of the physics of waves in a compressible medium --- the
same medium that forms light, matter, and ourselves.

\subsection{5.2 Why Motion Through the LCM Causes Length Contraction}

When relativity is taught using spacetime diagrams, length contraction
is often presented as a geometric effect: moving objects ``shrink''
along their direction of travel because of how coordinates transform.
While mathematically precise, that explanation can feel abstract and
disconnected from physical intuition.

In the 2MM picture, length contraction emerges far more naturally. It is
simply what happens when an object moves rapidly through a \textbf{real, compressible medium}: the LCM.

As an object accelerates, it must push its way through the LCM. At low
speeds this has almost no noticeable effect, but as the speed approaches
the propagation speed of transverse waves in the LCM (what we call the
speed of light), something important happens: the LCM begins to
\textbf{pile up} in front of the object.

This buildup creates a \textbf{directional compression gradient}:

\begin{itemize}
\tightlist
\item
  \textbf{Higher LCM density ahead of the object}
\item
  \textbf{Normal LCM density behind it}
\end{itemize}

This gradient grows steeper the faster the object moves. And because all
matter in 2MM consists of standing waves stabilized by the surrounding
LCM, a denser medium in front exerts a real, physical inward pressure on
the wave structure.

Crucially, there is \textbf{no pulling from behind}. The LCM behind the
object is simply less compressed; it does not contribute actively to the
contraction. Instead, it fails to counterbalance the inward pressure
from the front. The net effect is a \textbf{one-sided squeezing force}
along the direction of motion.

From an outside observer's perspective, the object is genuinely
shortened --- its entire structure is compressed along the direction of
travel by the piling up of LCM ahead of it.

But from the inside, nothing appears distorted.

All internal rulers, atoms, and even the biochemical processes that
constitute perception are made of the same standing waves immersed in
the same LCM. When the medium compresses in one direction, \textbf{every component of the object compresses together}. No internal measurement
reveals the contraction because the measuring instruments themselves
contract by the same proportion.

This perspective makes length contraction feel intuitive:

\begin{quote}
\textbf{Length contraction happens because fast motion through the LCM creates a directional compression gradient, and all matter --- being made of LCM standing waves --- must adapt to the compressed medium.}
\end{quote}

There is no paradox, no mystical geometric effect, and no need to
imagine ``shrinking'' as an illusion. It is the straightforward
mechanical response of waves interacting with a real medium.

\subsection{5.3 Mutual Causation in the Two Media}

Any elastic-aether model naturally interprets spacetime curvature as
variations in the density or tension of the medium itself. What 2MM adds
is a dynamic role for gravity in \emph{maintaining} that curvature. In
the geometric view of relativity, the sequence seems straightforward:
matter and energy bend spacetime, and gravity is the motion of objects
through that curvature. In 2MM the picture becomes more intertwined. A
standing wave \emph{is} a region of denser LCM, yet that denser region
can only remain stable because the GCM flux continually reinforces it
through shadowing. At the same time, the GCM would have nothing to
shadow if not for the density created by the standing wave. The result
is a self-sustaining feedback loop between the two media---neither the
LCM nor GCM ``comes first,'' and no initial moment is required. The
system simply persists, with curvature, matter, and gravity all emerging
from their mutual interaction.

This interdependence between the two media leads to an intriguing
implication: for 2MM to function as described, the system cannot easily
accommodate a true ``beginning.'' A beginning would require one medium
to exist and operate without the other, or for standing waves to form
before the shadowing effects that sustain them --- both of which break
the causal loops that the model depends on. In fact, any notion of an
absolute beginning already strains causality in conventional physics as
well. In 2MM, this tension resolves naturally: the universe is not
something that starts and then evolves; it is something that
\emph{self-maintains}, with structure emerging continually from the
ongoing reciprocity between the LCM and GCM.

\subsection{5.4 Absolute Time, the GCM, and Faster-Than-Light Transmission}

In 2MM, the presence of the GCM introduces a deeper notion of
\textbf{absolute time}---a timebase that exists independently of the LCM
processes that define our clocks and our experience. All measurement,
sensation, and physical activity in living systems occurs through
LCM-standing-wave structures, so we have access only to \emph{LCM time}.
But the GCM operates on its own timescale. It is not constrained by the
wave-propagation speed of the LCM (the speed of light), nor by the
relativistic limits that emerge from LCM-based processes.

This idea aligns with Tom Van Flandern's careful analysis in
\textbf{``\href{https://www.metaresearch.org/cosmology/cosmology2/the-speed-of-gravity-what-the-experiments-say}{The Speed of Gravity: What the Experiments Say}'' (1998)}, where he reviewed
timing delays, planetary motions, and signal propagation. His conclusion
was that gravity-like influences must propagate extraordinarily fast:

\begin{quote}
\textbf{Gravity must exceed 2 x 10\(^10\) c}, at least twenty billion times
the speed of light.
\end{quote}

In the context of 2MM, this is exactly the behavior expected of the GCM
flux. It moves on an inaccessible absolute timebase and is free to
propagate vastly faster than light. This immediately suggests that
\textbf{faster-than-light transmission} is physically possible if a
civilization can learn to encode patterns into the GCM.

But this does \textbf{not} permit faster-than-light communication in the
usual sense. The distinction is fundamental:

\begin{itemize}
\tightlist
\item
  \textbf{Transmission} is the propagation of patterns through the GCM.
\item
  \textbf{Communication} requires encoding and decoding, both of which
  happen in the LCM.
\end{itemize}

Encoding information into the GCM, and reading it out on the other end,
requires ordinary physical processes---electronics, detectors,
biological cognition---all of which are constrained by LCM time and the
speed of light. So even if a signal propagates nearly instantaneously
through the GCM, the total round-trip communication time cannot exceed
relativistic limits.

Thus:

\begin{quote}
\textbf{FTL transmission is possible in theory. FTL round-trip communication is not.}
\end{quote}

This preserves causality. No paradoxes arise because observers made of
LCM-standing-wave matter can never process a complete exchange faster
than light allows.

Yet the implications are profound. A civilization could, in principle,
carry on a \textbf{real-time conversation between Earth and Proxima Centauri}, because the GCM would carry the signal almost instantly. The
subjective experience would feel simultaneous, even though the
underlying encoding and decoding steps remain light-speed bound.

\paragraph{Implication for SETI}

This also leads to a natural expectation: if technological civilizations
develop a way to use the GCM for signaling, then \textbf{electromagnetic communication becomes obsolete}. EM transmission is slow, lossy, and
limited by LCM constraints, whereas the GCM allows nearly instantaneous
signaling over interstellar distances.

Thus, 2MM implies that \textbf{SETI's strategy of searching the electromagnetic spectrum may never detect advanced civilizations}, not
because they do not exist, but because their communication technology
would likely move beyond EM channels entirely. They would use the
GCM---an invisible, ultra-fast substrate that our current instruments
cannot yet manipulate or detect.

In this light, SETI's silence may be a clue rather than a contradiction:
advanced civilizations might be speaking in a medium we are not yet
listening to.

\subsection{5.5 Oumuamua as an Illustration of GCM Shielding (Speculative)}

`Oumuamua's unusual properties---its extreme elongation, lack of
outgassing, and complex tumbling---have made it a persistent point of
interest in the literature. Natural origins remain entirely plausible.
However, within the 2MM framework, the object provides a useful example
for visualizing how \textbf{GCM shielding} behaves for elongated bodies.

In LeSage-style theories (and in 2MM's refinement of them), sufficiently
dense material can act as a \textbf{gravity shield}: its interior
receives slightly less GCM flux than its surface because a portion of
the incoming flux is absorbed or scattered. This shielding effect is
tiny for ordinary matter, but even tiny differences can accumulate or
become directionally structured if the object has an extreme geometry.

For an elongated body like `Oumuamua, the degree of shielding is
\textbf{greater along its long axis} than along its short one. This
means the GCM flux deficit it produces---the \emph{shadow} cast into
downstream space---depends on orientation. As the object rotates, the
direction and amplitude of the shadow vary in a periodic way.

This creates a simple conceptual possibility:

\begin{quote}
A long, rotating object can act as a \textbf{passive gravitational beacon}, encoding a repeating pattern into the GCM flux purely by virtue
of its geometry and rotation.
\end{quote}

No technology is implied beyond the shape itself. No propulsion
mechanism is needed. No electromagnetic emission is required. The effect
arises naturally if an elongated mass rotates within a pervasive GCM
flux.

This does \textbf{not} suggest that `Oumuamua \emph{is} such a beacon.
But its geometry makes it a convenient illustration of how, in a
dual-medium ontology, even simple rotating shapes can produce
\textbf{directionally varying GCM signatures} --- signatures that would
be invisible electromagnetically but potentially detectable in a theory
where GCM shielding plays a meaningful role.
