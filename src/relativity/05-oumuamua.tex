\subsection{Oumuamua as an Illustration of GCM Shielding (Speculative)}

`Oumuamua's unusual properties---its extreme elongation, lack of
outgassing, and complex tumbling---have made it a persistent point of
interest in the literature. Natural origins remain entirely plausible.
However, within the 2MM framework, the object provides a useful example
for visualizing how \textbf{GCM shielding} behaves for elongated bodies.

In LeSage-style theories (and in 2MM's refinement of them), sufficiently
dense material can act as a \textbf{gravity shield}: its interior
receives slightly less GCM flux than its surface because a portion of
the incoming flux is absorbed or scattered. This shielding effect is
tiny for ordinary matter, but even tiny differences can accumulate or
become directionally structured if the object has an extreme geometry.

For an elongated body like `Oumuamua, the degree of shielding is
\textbf{greater along its long axis} than along its short one. This
means the GCM flux deficit it produces---the \emph{shadow} cast into
downstream space---depends on orientation. As the object rotates, the
direction and amplitude of the shadow vary in a periodic way.

This creates a simple conceptual possibility:

\begin{quote}
A long, rotating object can act as a \textbf{passive gravitational beacon}, encoding a repeating pattern into the GCM flux purely by virtue
of its geometry and rotation.
\end{quote}

No technology is implied beyond the shape itself. No propulsion
mechanism is needed. No electromagnetic emission is required. The effect
arises naturally if an elongated mass rotates within a pervasive GCM
flux.

This does \textbf{not} suggest that `Oumuamua \emph{is} such a beacon.
But its geometry makes it a convenient illustration of how, in a
dual-medium ontology, even simple rotating shapes can produce
\textbf{directionally varying GCM signatures} --- signatures that would
be invisible electromagnetically but potentially detectable in a theory
where GCM shielding plays a meaningful role.
