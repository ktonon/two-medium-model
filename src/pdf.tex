\input{preamble}
\usepackage[style=numeric, sorting=none]{biblatex}
\addbibresource{references.bib}

\title{The Two-Medium Model (2MM):\\
A Narrative Framework for Matter, Gravity, Light, and Cosmology}

\author{Kevin Tonon}
\date{\today}

\begin{document}
\maketitle
\begin{center}
  \url{https://github.com/ktonon/two-medium-model}
\end{center}

\begin{abstract}
The Two-Medium Model (2MM) describes the observable universe in terms of the interaction between two pervasive substrates: \textit{(i)} the \textbf{Light-Carrying Medium (LCM)}, an elastic continuum supporting compression, shear, and torsional distortions—including the standing-wave structures associated with matter; and \textit{(ii)} the \textbf{Gravity-Carrying Medium (GCM)}, a flux of ultra-small, ultra-fast corpuscles whose flow is partially obstructed by dense regions of the LCM, giving rise to gravitational attraction through momentum-flux shadowing.

Matter arises as standing waves in the LCM stabilized by GCM flux.
Gravity results from momentum imbalance in the GCM. Electrostatic,
electromagnetic, and nuclear binding forces all arise from the dynamic
balance maintained between the two media. Cosmological
redshift arises from the gradual, non-scattering loss of energy
from LCM travelling waves into the GCM. The CMB is the accumulated
remnant of this process in an infinite universe. The conceptual origin of 2MM
lies in the Meta Model, which proposed that two interacting substrates
are necessary to capture the richness of physical phenomena. This
framework, while qualitative, offers a physically intuitive ontology
unifying microphysics and cosmology and suggests avenues for future
formalization and testing.
\end{abstract}

{
\setcounter{tocdepth}{3}
\tableofcontents
}

\section{Preface}

This document is an attempt to answer a question that has interested me
for many years: \emph{Is it possible to build a coherent, medium-based picture of physics that connects phenomena across scales, from particles to cosmology, using simple and intuitive principles?}

I approached this project not as a professional physicist but as an
amateur who has spent decades reading about science, puzzles, and
anomalies that invite deeper questioning. My interest has always leaned
toward the edges of what we understand: places where standard
explanations work mathematically but remain conceptually opaque, or
where observations fit uneasily into the prevailing frameworks.

The ideas presented here began with a few philosophical commitments:
that interactions should be local, that waves require a medium, and that
forces should have physical carriers rather than operating at a
distance. These assumptions are not universally accepted, and they are
not proposed as corrections to modern physics. They simply provided a
different starting point, one that led me to explore whether a
medium-based ontology might unify many disconnected pieces of physical
behavior.

This is not a finished theory. It is a conceptual model: speculative,
exploratory, and intentionally not defined mathematically. My hope is
that general readers find it accessible, and that professionals find it
interesting enough to question, critique, or even refine. Science
progresses not only through detailed calculation, but also through new
ways of seeing familiar things.

\section{Introduction}

This document explores whether a coherent, medium-based picture of physics can be constructed that connects phenomena across scales, from subatomic behavior to cosmology, using simple and physically intuitive principles. Rather than proposing a formal theory, it presents a conceptual framework intended to clarify mechanisms that are often treated abstractly in modern physics.

Contemporary physical theories are extraordinarily successful at prediction, yet they frequently remain noncommittal about ontology. Fields are defined mathematically, forces emerge from operators or curvature, and fundamental particles are characterized by properties rather than internal structure. While effective, this approach leaves open the question of whether a more explicitly physical, mechanism-based description might also exist—one that preserves empirical success while offering greater geometric and intuitive coherence.

The Two-Medium Model (2MM) begins from a small set of foundational assumptions: that interactions are local, that waves require a supporting medium, and that gravity must be physically mediated rather than action-at-a-distance. These assumptions are neither new nor universally accepted, but they provide a consistent starting point for constructing an alternative conceptual picture grounded in observed phenomena.

From these premises, the model introduces two interacting components: a compressible medium that supports wave motion (LCM), and a high-speed particulate flux that gives rise to gravitational effects through shadowing and pressure imbalance (GCM). Within this framework, matter, forces, and large-scale structures are interpreted as emergent outcomes of medium interactions, wave confinement, and energy exchange between these two components.

This paper summarizes the resulting framework in a structured, narrative form. It does not attempt to reproduce the mathematical precision of quantum field theory or general relativity, but instead aims to provide an internally coherent conceptual model that naturally connects a wide range of phenomena, including electric charge, magnetism, nuclear structure, planetary mass evolution, redshift, and the behavior of active galactic nuclei. Some interpretations align with conventional thinking, while others depart from it.

The sections that follow are intended to be read sequentially, as the model builds cumulatively: particles emerge from wave confinement, forces from medium interactions, cosmological behavior from matter and energy cycling, and planetary-scale phenomena from long-term medium dynamics. The goal is not to assert correctness, but to explore what becomes possible when a different set of foundational assumptions is adopted and followed consistently.

\subsection{Predictions}

Although 2MM is not defined mathematically, it does make several
qualitative predictions that differ from standard interpretations and
could, in principle, be tested. These predictions arise directly from
the model's internal mechanisms: wave confinement, medium interactions,
particle structure, and equilibrium cycles.

\begin{enumerate}
\tightlist
\item
  \textbf{\hyperref[earth-and-the-solar-system]{Hydrogen creation in planets and moons}} 
  \textit{— and liquid water worlds should be abundant}
\item
  \textbf{\hyperref[voids-as-repulsive-regions]{Cosmic voids as repulsive regions}}
  \textit{— as part of energy cycling}
\item
  \textbf{\hyperref[neutron-formation-and-release-in-agn-cores]{Neutron deconfinement in AGN cores}}
  \textit{— as part of matter cycling}
\item
  \textbf{\hyperref[absolute-time]{SETI is unlikely to succeed}}
  \textit{— even if advanced civilizations exist}
\end{enumerate}

Each of these will be discussed in greater detail in relevant sections
below.



\section{Core Ontology: Two Interacting Media}

The Two-Medium Model begins with a simple idea: the physical world
arises from the interaction of two different substrates. One medium
carries waves and can form standing-wave particles. The other supplies a
fast-moving background flux that shapes how those waves behave and how
particles influence one another. Most of the complexity of physics, in
this picture, comes not from many different forces, but from the way
these two media continually feed back on each other.

\subsection{The LCM: A Medium for Waves and Matter}

The Light-Carrying Medium (LCM) is a compressible elastic substance that supports compression, shear, and torsional distortions. Traveling waves with phase-locked shear and torsional oscillations correspond to what we ordinarily call light.

What makes the LCM especially important is how it stores and manipulates
energy. It can hold energy as density gradients, torsional motion, or
shear patterns, and under extreme conditions it can reorganize traveling
waves into localized standing structures. These standing waves become
what we recognize as matter. Because the LCM deforms around these
structures, it also shapes how the GCM flux interacts with them,
indirectly setting the stage for every familiar force.

The LCM does not independently ``produce'' electric or magnetic fields.
Instead, the deformations it carries---compression patterns, twists, and
gradients---provide half of the mechanism that later becomes
electrostatic, magnetic, and nuclear behavior. In this model, electric
and magnetic fields are simply the ways compressed or twisted regions of
the LCM guide the flow of the GCM.
 
\subsection{The GCM: A Fast Flux That Generates Forces}

The Gravity-Carrying Medium (GCM) is a very different kind of substrate.
It consists of an ultra-fast, high-flux stream of tiny particles moving
through space. When these particles pass through matter, a small
fraction of them interact in slightly inelastic ways. These tiny
encounters create ``shadows'' in the flux, which show up as the force we
call gravity.

The strength of a shadow depends on the structure of the LCM around the
particle. A tightly compressed standing wave blocks or redirects more of
the GCM than a loose one, so momentum flows inward more strongly. In
this way, the GCM does not just produce gravity---it participates in
everything from the cohesion of atomic nuclei to the subtle repulsive
and attractive forces that arise between charged particles.

Crucially, the GCM and LCM are deeply interdependent. The GCM shapes the
LCM by pushing on it, compressing it, or twisting it around standing
waves. And the LCM, in return, determines how much of the GCM penetrates
or is scattered by a region. Together, they create the entire landscape
of forces we observe. Later sections discuss how this interaction also
avoids the overheating problem that plagued earlier LeSage-type
theories.

In this ontology, the GCM provides the \textbf{momentum-transfer component}
of all physical interactions, while the LCM provides the
\textbf{geometric and structural component}. What we call ``forces''
emerge from the interplay between the two.

Although the individual ingredients of this framework---the idea of a
wave-supporting medium, a momentum-carrying flux, and standing waves as
particles---each have historical precedents, their combination into a
single interacting system is what gives 2MM its explanatory power. By
treating the LCM and GCM as mutually shaping components of one physical
ontology, the model brings together ideas that were previously separate
and shows how they can work together to produce the full range of
observed phenomena.

\subsection{Very Different Scales: Why Two Media Can Coexist}

A key feature of 2MM is that the two media do not operate at the same
scale. The LCM behaves like a smooth, continuous medium at the scales
relevant to light and matter. The GCM, by contrast, is made of extremely
small, extremely fast particles. This large separation in scale is what
allows both media to occupy the same space without simply behaving like
one blended substance.

In the spirit of Tom Van Flandern's meta-model, the GCM particles are
taken to be \textbf{far smaller than any length scale we can currently probe},
effectively much smaller than the Planck length for practical
purposes. Their role is not to form visible structure, but to provide a
nearly uniform, ultra-fine background flux that transfers momentum.
Because these particles are so small and so numerous, they can stream
through the LCM with almost no disturbance, except in regions where the
LCM is highly compressed or organized into standing waves.

The GCM flux is also assumed to move \textbf{much faster than light}.
This idea, again inspired by TVF's arguments for a superluminal gravity
medium, allows gravitational effects (in the shadowing sense) to
propagate effectively instantaneously at the scales we observe, without
conflicting with the observed behavior of light. In 2MM, light is
limited by the properties of the LCM, while the GCM operates on a
deeper, faster layer.

This separation of scale and speed has three important consequences:

\begin{itemize}
\tightlist
\item
  The \textbf{LCM} can carry waves and form standing-wave particles
  without being torn apart by the GCM.
\item
  The \textbf{GCM} can provide a persistent, high-speed momentum flux
  that responds sensitively to LCM compression and standing-wave
  structure.
\item
  Together, they can generate complex behaviors: gravity, fields, and
  forces; without requiring either medium alone to ``do everything.''
\end{itemize}

2MM builds on these scale assumptions and extends them: the GCM is not
only a gravity carrier, but also the mechanism that confines standing
waves, maintains particle compression, and couples to LCM deformations
to produce all familiar interactions. The extreme smallness and high
speed of the GCM particles are what make it possible for the two media
to coexist and yet play very different physical roles.

\section{Core Ontology: Two Interacting Media}

The Two-Medium Model begins with a simple idea: the physical world
arises from the interaction of two different substrates. One medium
carries waves and can form standing-wave particles. The other supplies a
fast-moving background flux that shapes how those waves behave and how
particles influence one another. Most of the complexity of physics, in
this picture, comes not from many different forces, but from the way
these two media continually feed back on each other.

\subsection{The LCM: A Medium for Waves and Matter}

The Light-Carrying Medium (LCM) is a compressible elastic substance that supports compression, shear, and torsional distortions. Traveling waves with phase-locked shear and torsional oscillations correspond to what we ordinarily call light.

What makes the LCM especially important is how it stores and manipulates
energy. It can hold energy as density gradients, torsional motion, or
shear patterns, and under extreme conditions it can reorganize traveling
waves into localized standing structures. These standing waves become
what we recognize as matter. Because the LCM deforms around these
structures, it also shapes how the GCM flux interacts with them,
indirectly setting the stage for every familiar force.

The LCM does not independently ``produce'' electric or magnetic fields.
Instead, the deformations it carries---compression patterns, twists, and
gradients---provide half of the mechanism that later becomes
electrostatic, magnetic, and nuclear behavior. In this model, electric
and magnetic fields are simply the ways compressed or twisted regions of
the LCM guide the flow of the GCM.
 
\subsection{The GCM: A Fast Flux That Generates Forces}

The Gravity-Carrying Medium (GCM) is a very different kind of substrate.
It consists of an ultra-fast, high-flux stream of tiny particles moving
through space. When these particles pass through matter, a small
fraction of them interact in slightly inelastic ways. These tiny
encounters create ``shadows'' in the flux, which show up as the force we
call gravity.

The strength of a shadow depends on the structure of the LCM around the
particle. A tightly compressed standing wave blocks or redirects more of
the GCM than a loose one, so momentum flows inward more strongly. In
this way, the GCM does not just produce gravity---it participates in
everything from the cohesion of atomic nuclei to the subtle repulsive
and attractive forces that arise between charged particles.

Crucially, the GCM and LCM are deeply interdependent. The GCM shapes the
LCM by pushing on it, compressing it, or twisting it around standing
waves. And the LCM, in return, determines how much of the GCM penetrates
or is scattered by a region. Together, they create the entire landscape
of forces we observe. Later sections discuss how this interaction also
avoids the overheating problem that plagued earlier LeSage-type
theories.

In this ontology, the GCM provides the \textbf{momentum-transfer component}
of all physical interactions, while the LCM provides the
\textbf{geometric and structural component}. What we call ``forces''
emerge from the interplay between the two.

Although the individual ingredients of this framework---the idea of a
wave-supporting medium, a momentum-carrying flux, and standing waves as
particles---each have historical precedents, their combination into a
single interacting system is what gives 2MM its explanatory power. By
treating the LCM and GCM as mutually shaping components of one physical
ontology, the model brings together ideas that were previously separate
and shows how they can work together to produce the full range of
observed phenomena.

\subsection{Very Different Scales: Why Two Media Can Coexist}

A key feature of 2MM is that the two media do not operate at the same
scale. The LCM behaves like a smooth, continuous medium at the scales
relevant to light and matter. The GCM, by contrast, is made of extremely
small, extremely fast particles. This large separation in scale is what
allows both media to occupy the same space without simply behaving like
one blended substance.

In the spirit of Tom Van Flandern's meta-model, the GCM particles are
taken to be \textbf{far smaller than any length scale we can currently probe},
effectively much smaller than the Planck length for practical
purposes. Their role is not to form visible structure, but to provide a
nearly uniform, ultra-fine background flux that transfers momentum.
Because these particles are so small and so numerous, they can stream
through the LCM with almost no disturbance, except in regions where the
LCM is highly compressed or organized into standing waves.

The GCM flux is also assumed to move \textbf{much faster than light}.
This idea, again inspired by TVF's arguments for a superluminal gravity
medium, allows gravitational effects (in the shadowing sense) to
propagate effectively instantaneously at the scales we observe, without
conflicting with the observed behavior of light. In 2MM, light is
limited by the properties of the LCM, while the GCM operates on a
deeper, faster layer.

This separation of scale and speed has three important consequences:

\begin{itemize}
\tightlist
\item
  The \textbf{LCM} can carry waves and form standing-wave particles
  without being torn apart by the GCM.
\item
  The \textbf{GCM} can provide a persistent, high-speed momentum flux
  that responds sensitively to LCM compression and standing-wave
  structure.
\item
  Together, they can generate complex behaviors: gravity, fields, and
  forces; without requiring either medium alone to ``do everything.''
\end{itemize}

2MM builds on these scale assumptions and extends them: the GCM is not
only a gravity carrier, but also the mechanism that confines standing
waves, maintains particle compression, and couples to LCM deformations
to produce all familiar interactions. The extreme smallness and high
speed of the GCM particles are what make it possible for the two media
to coexist and yet play very different physical roles.

\section{Core Ontology: Two Interacting Media}

The Two-Medium Model begins with a simple idea: the physical world
arises from the interaction of two different substrates. One medium
carries waves and can form standing-wave particles. The other supplies a
fast-moving background flux that shapes how those waves behave and how
particles influence one another. Most of the complexity of physics, in
this picture, comes not from many different forces, but from the way
these two media continually feed back on each other.

\subsection{The LCM: A Medium for Waves and Matter}

The Light-Carrying Medium (LCM) is a compressible elastic substance that supports compression, shear, and torsional distortions. Traveling waves with phase-locked shear and torsional oscillations correspond to what we ordinarily call light.

What makes the LCM especially important is how it stores and manipulates
energy. It can hold energy as density gradients, torsional motion, or
shear patterns, and under extreme conditions it can reorganize traveling
waves into localized standing structures. These standing waves become
what we recognize as matter. Because the LCM deforms around these
structures, it also shapes how the GCM flux interacts with them,
indirectly setting the stage for every familiar force.

The LCM does not independently ``produce'' electric or magnetic fields.
Instead, the deformations it carries---compression patterns, twists, and
gradients---provide half of the mechanism that later becomes
electrostatic, magnetic, and nuclear behavior. In this model, electric
and magnetic fields are simply the ways compressed or twisted regions of
the LCM guide the flow of the GCM.
 
\subsection{The GCM: A Fast Flux That Generates Forces}

The Gravity-Carrying Medium (GCM) is a very different kind of substrate.
It consists of an ultra-fast, high-flux stream of tiny particles moving
through space. When these particles pass through matter, a small
fraction of them interact in slightly inelastic ways. These tiny
encounters create ``shadows'' in the flux, which show up as the force we
call gravity.

The strength of a shadow depends on the structure of the LCM around the
particle. A tightly compressed standing wave blocks or redirects more of
the GCM than a loose one, so momentum flows inward more strongly. In
this way, the GCM does not just produce gravity---it participates in
everything from the cohesion of atomic nuclei to the subtle repulsive
and attractive forces that arise between charged particles.

Crucially, the GCM and LCM are deeply interdependent. The GCM shapes the
LCM by pushing on it, compressing it, or twisting it around standing
waves. And the LCM, in return, determines how much of the GCM penetrates
or is scattered by a region. Together, they create the entire landscape
of forces we observe. Later sections discuss how this interaction also
avoids the overheating problem that plagued earlier LeSage-type
theories.

In this ontology, the GCM provides the \textbf{momentum-transfer component}
of all physical interactions, while the LCM provides the
\textbf{geometric and structural component}. What we call ``forces''
emerge from the interplay between the two.

Although the individual ingredients of this framework---the idea of a
wave-supporting medium, a momentum-carrying flux, and standing waves as
particles---each have historical precedents, their combination into a
single interacting system is what gives 2MM its explanatory power. By
treating the LCM and GCM as mutually shaping components of one physical
ontology, the model brings together ideas that were previously separate
and shows how they can work together to produce the full range of
observed phenomena.

\subsection{Very Different Scales: Why Two Media Can Coexist}

A key feature of 2MM is that the two media do not operate at the same
scale. The LCM behaves like a smooth, continuous medium at the scales
relevant to light and matter. The GCM, by contrast, is made of extremely
small, extremely fast particles. This large separation in scale is what
allows both media to occupy the same space without simply behaving like
one blended substance.

In the spirit of Tom Van Flandern's meta-model, the GCM particles are
taken to be \textbf{far smaller than any length scale we can currently probe},
effectively much smaller than the Planck length for practical
purposes. Their role is not to form visible structure, but to provide a
nearly uniform, ultra-fine background flux that transfers momentum.
Because these particles are so small and so numerous, they can stream
through the LCM with almost no disturbance, except in regions where the
LCM is highly compressed or organized into standing waves.

The GCM flux is also assumed to move \textbf{much faster than light}.
This idea, again inspired by TVF's arguments for a superluminal gravity
medium, allows gravitational effects (in the shadowing sense) to
propagate effectively instantaneously at the scales we observe, without
conflicting with the observed behavior of light. In 2MM, light is
limited by the properties of the LCM, while the GCM operates on a
deeper, faster layer.

This separation of scale and speed has three important consequences:

\begin{itemize}
\tightlist
\item
  The \textbf{LCM} can carry waves and form standing-wave particles
  without being torn apart by the GCM.
\item
  The \textbf{GCM} can provide a persistent, high-speed momentum flux
  that responds sensitively to LCM compression and standing-wave
  structure.
\item
  Together, they can generate complex behaviors: gravity, fields, and
  forces; without requiring either medium alone to ``do everything.''
\end{itemize}

2MM builds on these scale assumptions and extends them: the GCM is not
only a gravity carrier, but also the mechanism that confines standing
waves, maintains particle compression, and couples to LCM deformations
to produce all familiar interactions. The extreme smallness and high
speed of the GCM particles are what make it possible for the two media
to coexist and yet play very different physical roles.

\section{Core Ontology: Two Interacting Media}

The Two-Medium Model begins with a simple idea: the physical world
arises from the interaction of two different substrates. One medium
carries waves and can form standing-wave particles. The other supplies a
fast-moving background flux that shapes how those waves behave and how
particles influence one another. Most of the complexity of physics, in
this picture, comes not from many different forces, but from the way
these two media continually feed back on each other.

\subsection{The LCM: A Medium for Waves and Matter}

The Light-Carrying Medium (LCM) is a compressible elastic substance that supports compression, shear, and torsional distortions. Traveling waves with phase-locked shear and torsional oscillations correspond to what we ordinarily call light.

What makes the LCM especially important is how it stores and manipulates
energy. It can hold energy as density gradients, torsional motion, or
shear patterns, and under extreme conditions it can reorganize traveling
waves into localized standing structures. These standing waves become
what we recognize as matter. Because the LCM deforms around these
structures, it also shapes how the GCM flux interacts with them,
indirectly setting the stage for every familiar force.

The LCM does not independently ``produce'' electric or magnetic fields.
Instead, the deformations it carries---compression patterns, twists, and
gradients---provide half of the mechanism that later becomes
electrostatic, magnetic, and nuclear behavior. In this model, electric
and magnetic fields are simply the ways compressed or twisted regions of
the LCM guide the flow of the GCM.
 
\subsection{The GCM: A Fast Flux That Generates Forces}

The Gravity-Carrying Medium (GCM) is a very different kind of substrate.
It consists of an ultra-fast, high-flux stream of tiny particles moving
through space. When these particles pass through matter, a small
fraction of them interact in slightly inelastic ways. These tiny
encounters create ``shadows'' in the flux, which show up as the force we
call gravity.

The strength of a shadow depends on the structure of the LCM around the
particle. A tightly compressed standing wave blocks or redirects more of
the GCM than a loose one, so momentum flows inward more strongly. In
this way, the GCM does not just produce gravity---it participates in
everything from the cohesion of atomic nuclei to the subtle repulsive
and attractive forces that arise between charged particles.

Crucially, the GCM and LCM are deeply interdependent. The GCM shapes the
LCM by pushing on it, compressing it, or twisting it around standing
waves. And the LCM, in return, determines how much of the GCM penetrates
or is scattered by a region. Together, they create the entire landscape
of forces we observe. Later sections discuss how this interaction also
avoids the overheating problem that plagued earlier LeSage-type
theories.

In this ontology, the GCM provides the \textbf{momentum-transfer component}
of all physical interactions, while the LCM provides the
\textbf{geometric and structural component}. What we call ``forces''
emerge from the interplay between the two.

Although the individual ingredients of this framework---the idea of a
wave-supporting medium, a momentum-carrying flux, and standing waves as
particles---each have historical precedents, their combination into a
single interacting system is what gives 2MM its explanatory power. By
treating the LCM and GCM as mutually shaping components of one physical
ontology, the model brings together ideas that were previously separate
and shows how they can work together to produce the full range of
observed phenomena.

\subsection{Very Different Scales: Why Two Media Can Coexist}

A key feature of 2MM is that the two media do not operate at the same
scale. The LCM behaves like a smooth, continuous medium at the scales
relevant to light and matter. The GCM, by contrast, is made of extremely
small, extremely fast particles. This large separation in scale is what
allows both media to occupy the same space without simply behaving like
one blended substance.

In the spirit of Tom Van Flandern's meta-model, the GCM particles are
taken to be \textbf{far smaller than any length scale we can currently probe},
effectively much smaller than the Planck length for practical
purposes. Their role is not to form visible structure, but to provide a
nearly uniform, ultra-fine background flux that transfers momentum.
Because these particles are so small and so numerous, they can stream
through the LCM with almost no disturbance, except in regions where the
LCM is highly compressed or organized into standing waves.

The GCM flux is also assumed to move \textbf{much faster than light}.
This idea, again inspired by TVF's arguments for a superluminal gravity
medium, allows gravitational effects (in the shadowing sense) to
propagate effectively instantaneously at the scales we observe, without
conflicting with the observed behavior of light. In 2MM, light is
limited by the properties of the LCM, while the GCM operates on a
deeper, faster layer.

This separation of scale and speed has three important consequences:

\begin{itemize}
\tightlist
\item
  The \textbf{LCM} can carry waves and form standing-wave particles
  without being torn apart by the GCM.
\item
  The \textbf{GCM} can provide a persistent, high-speed momentum flux
  that responds sensitively to LCM compression and standing-wave
  structure.
\item
  Together, they can generate complex behaviors: gravity, fields, and
  forces; without requiring either medium alone to ``do everything.''
\end{itemize}

2MM builds on these scale assumptions and extends them: the GCM is not
only a gravity carrier, but also the mechanism that confines standing
waves, maintains particle compression, and couples to LCM deformations
to produce all familiar interactions. The extreme smallness and high
speed of the GCM particles are what make it possible for the two media
to coexist and yet play very different physical roles.

\section{Core Ontology: Two Interacting Media}

The Two-Medium Model begins with a simple idea: the physical world
arises from the interaction of two different substrates. One medium
carries waves and can form standing-wave particles. The other supplies a
fast-moving background flux that shapes how those waves behave and how
particles influence one another. Most of the complexity of physics, in
this picture, comes not from many different forces, but from the way
these two media continually feed back on each other.

\subsection{The LCM: A Medium for Waves and Matter}

The Light-Carrying Medium (LCM) is a compressible elastic substance that supports compression, shear, and torsional distortions. Traveling waves with phase-locked shear and torsional oscillations correspond to what we ordinarily call light.

What makes the LCM especially important is how it stores and manipulates
energy. It can hold energy as density gradients, torsional motion, or
shear patterns, and under extreme conditions it can reorganize traveling
waves into localized standing structures. These standing waves become
what we recognize as matter. Because the LCM deforms around these
structures, it also shapes how the GCM flux interacts with them,
indirectly setting the stage for every familiar force.

The LCM does not independently ``produce'' electric or magnetic fields.
Instead, the deformations it carries---compression patterns, twists, and
gradients---provide half of the mechanism that later becomes
electrostatic, magnetic, and nuclear behavior. In this model, electric
and magnetic fields are simply the ways compressed or twisted regions of
the LCM guide the flow of the GCM.
 
\subsection{The GCM: A Fast Flux That Generates Forces}

The Gravity-Carrying Medium (GCM) is a very different kind of substrate.
It consists of an ultra-fast, high-flux stream of tiny particles moving
through space. When these particles pass through matter, a small
fraction of them interact in slightly inelastic ways. These tiny
encounters create ``shadows'' in the flux, which show up as the force we
call gravity.

The strength of a shadow depends on the structure of the LCM around the
particle. A tightly compressed standing wave blocks or redirects more of
the GCM than a loose one, so momentum flows inward more strongly. In
this way, the GCM does not just produce gravity---it participates in
everything from the cohesion of atomic nuclei to the subtle repulsive
and attractive forces that arise between charged particles.

Crucially, the GCM and LCM are deeply interdependent. The GCM shapes the
LCM by pushing on it, compressing it, or twisting it around standing
waves. And the LCM, in return, determines how much of the GCM penetrates
or is scattered by a region. Together, they create the entire landscape
of forces we observe. Later sections discuss how this interaction also
avoids the overheating problem that plagued earlier LeSage-type
theories.

In this ontology, the GCM provides the \textbf{momentum-transfer component}
of all physical interactions, while the LCM provides the
\textbf{geometric and structural component}. What we call ``forces''
emerge from the interplay between the two.

Although the individual ingredients of this framework---the idea of a
wave-supporting medium, a momentum-carrying flux, and standing waves as
particles---each have historical precedents, their combination into a
single interacting system is what gives 2MM its explanatory power. By
treating the LCM and GCM as mutually shaping components of one physical
ontology, the model brings together ideas that were previously separate
and shows how they can work together to produce the full range of
observed phenomena.

\subsection{Very Different Scales: Why Two Media Can Coexist}

A key feature of 2MM is that the two media do not operate at the same
scale. The LCM behaves like a smooth, continuous medium at the scales
relevant to light and matter. The GCM, by contrast, is made of extremely
small, extremely fast particles. This large separation in scale is what
allows both media to occupy the same space without simply behaving like
one blended substance.

In the spirit of Tom Van Flandern's meta-model, the GCM particles are
taken to be \textbf{far smaller than any length scale we can currently probe},
effectively much smaller than the Planck length for practical
purposes. Their role is not to form visible structure, but to provide a
nearly uniform, ultra-fine background flux that transfers momentum.
Because these particles are so small and so numerous, they can stream
through the LCM with almost no disturbance, except in regions where the
LCM is highly compressed or organized into standing waves.

The GCM flux is also assumed to move \textbf{much faster than light}.
This idea, again inspired by TVF's arguments for a superluminal gravity
medium, allows gravitational effects (in the shadowing sense) to
propagate effectively instantaneously at the scales we observe, without
conflicting with the observed behavior of light. In 2MM, light is
limited by the properties of the LCM, while the GCM operates on a
deeper, faster layer.

This separation of scale and speed has three important consequences:

\begin{itemize}
\tightlist
\item
  The \textbf{LCM} can carry waves and form standing-wave particles
  without being torn apart by the GCM.
\item
  The \textbf{GCM} can provide a persistent, high-speed momentum flux
  that responds sensitively to LCM compression and standing-wave
  structure.
\item
  Together, they can generate complex behaviors: gravity, fields, and
  forces; without requiring either medium alone to ``do everything.''
\end{itemize}

2MM builds on these scale assumptions and extends them: the GCM is not
only a gravity carrier, but also the mechanism that confines standing
waves, maintains particle compression, and couples to LCM deformations
to produce all familiar interactions. The extreme smallness and high
speed of the GCM particles are what make it possible for the two media
to coexist and yet play very different physical roles.

\section{Core Ontology: Two Interacting Media}

The Two-Medium Model begins with a simple idea: the physical world
arises from the interaction of two different substrates. One medium
carries waves and can form standing-wave particles. The other supplies a
fast-moving background flux that shapes how those waves behave and how
particles influence one another. Most of the complexity of physics, in
this picture, comes not from many different forces, but from the way
these two media continually feed back on each other.

\subsection{The LCM: A Medium for Waves and Matter}

The Light-Carrying Medium (LCM) is a compressible elastic substance that supports compression, shear, and torsional distortions. Traveling waves with phase-locked shear and torsional oscillations correspond to what we ordinarily call light.

What makes the LCM especially important is how it stores and manipulates
energy. It can hold energy as density gradients, torsional motion, or
shear patterns, and under extreme conditions it can reorganize traveling
waves into localized standing structures. These standing waves become
what we recognize as matter. Because the LCM deforms around these
structures, it also shapes how the GCM flux interacts with them,
indirectly setting the stage for every familiar force.

The LCM does not independently ``produce'' electric or magnetic fields.
Instead, the deformations it carries---compression patterns, twists, and
gradients---provide half of the mechanism that later becomes
electrostatic, magnetic, and nuclear behavior. In this model, electric
and magnetic fields are simply the ways compressed or twisted regions of
the LCM guide the flow of the GCM.
 
\subsection{The GCM: A Fast Flux That Generates Forces}

The Gravity-Carrying Medium (GCM) is a very different kind of substrate.
It consists of an ultra-fast, high-flux stream of tiny particles moving
through space. When these particles pass through matter, a small
fraction of them interact in slightly inelastic ways. These tiny
encounters create ``shadows'' in the flux, which show up as the force we
call gravity.

The strength of a shadow depends on the structure of the LCM around the
particle. A tightly compressed standing wave blocks or redirects more of
the GCM than a loose one, so momentum flows inward more strongly. In
this way, the GCM does not just produce gravity---it participates in
everything from the cohesion of atomic nuclei to the subtle repulsive
and attractive forces that arise between charged particles.

Crucially, the GCM and LCM are deeply interdependent. The GCM shapes the
LCM by pushing on it, compressing it, or twisting it around standing
waves. And the LCM, in return, determines how much of the GCM penetrates
or is scattered by a region. Together, they create the entire landscape
of forces we observe. Later sections discuss how this interaction also
avoids the overheating problem that plagued earlier LeSage-type
theories.

In this ontology, the GCM provides the \textbf{momentum-transfer component}
of all physical interactions, while the LCM provides the
\textbf{geometric and structural component}. What we call ``forces''
emerge from the interplay between the two.

Although the individual ingredients of this framework---the idea of a
wave-supporting medium, a momentum-carrying flux, and standing waves as
particles---each have historical precedents, their combination into a
single interacting system is what gives 2MM its explanatory power. By
treating the LCM and GCM as mutually shaping components of one physical
ontology, the model brings together ideas that were previously separate
and shows how they can work together to produce the full range of
observed phenomena.

\subsection{Very Different Scales: Why Two Media Can Coexist}

A key feature of 2MM is that the two media do not operate at the same
scale. The LCM behaves like a smooth, continuous medium at the scales
relevant to light and matter. The GCM, by contrast, is made of extremely
small, extremely fast particles. This large separation in scale is what
allows both media to occupy the same space without simply behaving like
one blended substance.

In the spirit of Tom Van Flandern's meta-model, the GCM particles are
taken to be \textbf{far smaller than any length scale we can currently probe},
effectively much smaller than the Planck length for practical
purposes. Their role is not to form visible structure, but to provide a
nearly uniform, ultra-fine background flux that transfers momentum.
Because these particles are so small and so numerous, they can stream
through the LCM with almost no disturbance, except in regions where the
LCM is highly compressed or organized into standing waves.

The GCM flux is also assumed to move \textbf{much faster than light}.
This idea, again inspired by TVF's arguments for a superluminal gravity
medium, allows gravitational effects (in the shadowing sense) to
propagate effectively instantaneously at the scales we observe, without
conflicting with the observed behavior of light. In 2MM, light is
limited by the properties of the LCM, while the GCM operates on a
deeper, faster layer.

This separation of scale and speed has three important consequences:

\begin{itemize}
\tightlist
\item
  The \textbf{LCM} can carry waves and form standing-wave particles
  without being torn apart by the GCM.
\item
  The \textbf{GCM} can provide a persistent, high-speed momentum flux
  that responds sensitively to LCM compression and standing-wave
  structure.
\item
  Together, they can generate complex behaviors: gravity, fields, and
  forces; without requiring either medium alone to ``do everything.''
\end{itemize}

2MM builds on these scale assumptions and extends them: the GCM is not
only a gravity carrier, but also the mechanism that confines standing
waves, maintains particle compression, and couples to LCM deformations
to produce all familiar interactions. The extreme smallness and high
speed of the GCM particles are what make it possible for the two media
to coexist and yet play very different physical roles.

\section{Core Ontology: Two Interacting Media}

The Two-Medium Model begins with a simple idea: the physical world
arises from the interaction of two different substrates. One medium
carries waves and can form standing-wave particles. The other supplies a
fast-moving background flux that shapes how those waves behave and how
particles influence one another. Most of the complexity of physics, in
this picture, comes not from many different forces, but from the way
these two media continually feed back on each other.

\subsection{The LCM: A Medium for Waves and Matter}

The Light-Carrying Medium (LCM) is a compressible elastic substance that supports compression, shear, and torsional distortions. Traveling waves with phase-locked shear and torsional oscillations correspond to what we ordinarily call light.

What makes the LCM especially important is how it stores and manipulates
energy. It can hold energy as density gradients, torsional motion, or
shear patterns, and under extreme conditions it can reorganize traveling
waves into localized standing structures. These standing waves become
what we recognize as matter. Because the LCM deforms around these
structures, it also shapes how the GCM flux interacts with them,
indirectly setting the stage for every familiar force.

The LCM does not independently ``produce'' electric or magnetic fields.
Instead, the deformations it carries---compression patterns, twists, and
gradients---provide half of the mechanism that later becomes
electrostatic, magnetic, and nuclear behavior. In this model, electric
and magnetic fields are simply the ways compressed or twisted regions of
the LCM guide the flow of the GCM.
 
\subsection{The GCM: A Fast Flux That Generates Forces}

The Gravity-Carrying Medium (GCM) is a very different kind of substrate.
It consists of an ultra-fast, high-flux stream of tiny particles moving
through space. When these particles pass through matter, a small
fraction of them interact in slightly inelastic ways. These tiny
encounters create ``shadows'' in the flux, which show up as the force we
call gravity.

The strength of a shadow depends on the structure of the LCM around the
particle. A tightly compressed standing wave blocks or redirects more of
the GCM than a loose one, so momentum flows inward more strongly. In
this way, the GCM does not just produce gravity---it participates in
everything from the cohesion of atomic nuclei to the subtle repulsive
and attractive forces that arise between charged particles.

Crucially, the GCM and LCM are deeply interdependent. The GCM shapes the
LCM by pushing on it, compressing it, or twisting it around standing
waves. And the LCM, in return, determines how much of the GCM penetrates
or is scattered by a region. Together, they create the entire landscape
of forces we observe. Later sections discuss how this interaction also
avoids the overheating problem that plagued earlier LeSage-type
theories.

In this ontology, the GCM provides the \textbf{momentum-transfer component}
of all physical interactions, while the LCM provides the
\textbf{geometric and structural component}. What we call ``forces''
emerge from the interplay between the two.

Although the individual ingredients of this framework---the idea of a
wave-supporting medium, a momentum-carrying flux, and standing waves as
particles---each have historical precedents, their combination into a
single interacting system is what gives 2MM its explanatory power. By
treating the LCM and GCM as mutually shaping components of one physical
ontology, the model brings together ideas that were previously separate
and shows how they can work together to produce the full range of
observed phenomena.

\subsection{Very Different Scales: Why Two Media Can Coexist}

A key feature of 2MM is that the two media do not operate at the same
scale. The LCM behaves like a smooth, continuous medium at the scales
relevant to light and matter. The GCM, by contrast, is made of extremely
small, extremely fast particles. This large separation in scale is what
allows both media to occupy the same space without simply behaving like
one blended substance.

In the spirit of Tom Van Flandern's meta-model, the GCM particles are
taken to be \textbf{far smaller than any length scale we can currently probe},
effectively much smaller than the Planck length for practical
purposes. Their role is not to form visible structure, but to provide a
nearly uniform, ultra-fine background flux that transfers momentum.
Because these particles are so small and so numerous, they can stream
through the LCM with almost no disturbance, except in regions where the
LCM is highly compressed or organized into standing waves.

The GCM flux is also assumed to move \textbf{much faster than light}.
This idea, again inspired by TVF's arguments for a superluminal gravity
medium, allows gravitational effects (in the shadowing sense) to
propagate effectively instantaneously at the scales we observe, without
conflicting with the observed behavior of light. In 2MM, light is
limited by the properties of the LCM, while the GCM operates on a
deeper, faster layer.

This separation of scale and speed has three important consequences:

\begin{itemize}
\tightlist
\item
  The \textbf{LCM} can carry waves and form standing-wave particles
  without being torn apart by the GCM.
\item
  The \textbf{GCM} can provide a persistent, high-speed momentum flux
  that responds sensitively to LCM compression and standing-wave
  structure.
\item
  Together, they can generate complex behaviors: gravity, fields, and
  forces; without requiring either medium alone to ``do everything.''
\end{itemize}

2MM builds on these scale assumptions and extends them: the GCM is not
only a gravity carrier, but also the mechanism that confines standing
waves, maintains particle compression, and couples to LCM deformations
to produce all familiar interactions. The extreme smallness and high
speed of the GCM particles are what make it possible for the two media
to coexist and yet play very different physical roles.

\section{Methodology and Acknowledgements}

The development of this model was highly iterative and heavily assisted
by modern AI tools. Without such tools, the model described here would
likely have taken months or years to assemble. As someone with a job, a
family, and limited hours to devote to speculative research, I would not
have been able to sustain the rapid cycle of hypothesis, critique, and
refinement that this project required. AI made that pace possible.

However, the role of AI requires careful qualification. These tools did
not ``invent'' the ideas presented here. Instead, they made the
collective knowledge of the scientific community navigable in a way that
was never previously accessible to individuals outside formal research
settings. During development, I would pose challenges and suggests
constraints, and the AI would respond with ideas, critiques,
counterexamples, or alternative possibilities drawn from patterns across
the literature it had been trained on. Most of these suggestions were
rejected. Only a small subset survived repeated rounds of filtering and
conceptual testing on my part.

A clear example of this collaborative dynamic is the dual-oscillation
model of the electron and positron. The idea of combining two
oscillatory modes with a phase offset was surfaced in dialogue with an
AI system. I would not have discovered that configuration unaided. But
the structural constraints that shaped it --- the insistence on mirror
symmetry, three-dimensional non-flippability, pair-production balance,
and geometric stability --- came from my own criteria. The AI generated
raw candidate ideas; I applied the conceptual standards that determined
whether they were tenable. The final standing-wave geometry emerged
through that interaction.

This dynamic applies throughout the project. The synthesis is my own.
But several individual components surfaced only because AI made it
possible to explore a wide conceptual space quickly. For that reason, I
want to acknowledge not only the role of AI but also the deeper lineage
behind it: the thousands of scientists, educators, writers, engineers,
and students whose work forms the substrate upon which such tools are
built. There is no practical way to identify every individual influence,
but their collective contribution underlies every AI-assisted iteration.

If you believe that any ideas in this document overlap with work you
have published, and you would like your contributions acknowledged or
referenced, please feel free to open an issue or submit a pull request.
I will review it and update the references accordingly.

Several strands of prior work influenced the direction of this model.
Halton Arp's empirical studies of redshift anomalies suggested that
large-scale cosmology might still contain unaddressed puzzles. Tom Van
Flandern's ``Meta-Model'' introduced the idea of multiple interacting
media, a conceptual seed that eventually grew into the Two-Medium
framework presented here \cite{MetaResearchStructureOfMatter2003}. Dr. Chantal Roth's modeling of an elastic
aether encouraged me to consider that empty space could contain more
structure than I had previously assumed. The collection of papers in
Pushing Gravity: New Perspectives on Le Sage's Theory of Gravitation,
edited and contributed to by researcher Matthew R. Edwards, introduced
me to Le Sage-type gravity models and the challenges they
entail---particularly the overheating problem. Matthew himself first
introduced me to Expansion Tectonics decades ago, when I was completing
my computer science major at the University of Toronto.

While these works are not mainstream, they demonstrated that alternative
ontologies can still be logically organized and empirically motivated.

I present this work not as a finished theory, but as an invitation: to
reconsider whether some aspects of physical law might gain clarity from
a different underlying ontology, and to explore whether the ideas
outlined here might be refined, challenged, or developed further by
those with greater expertise and technical resources.



\printbibliography

\end{document}
